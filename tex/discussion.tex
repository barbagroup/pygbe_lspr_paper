%!TEX root = ClementiCooperBarba2018.tex

Extending \pygbe to the LSPR biosensing application required considerable code 
modifications and added functionality. The results presented in the previous section 
offer evidence to build confidence on the suitability of the mathematical model 
and the correctness of the code.
The grid-convergence study with a nanosphere under a constant electric field 
shows a $1/N$ rate of convergence, consistent with convergence results 
in previous work using \pygbe \cite{CooperBardhanBarba2013}.
Further verification of \pygbe's new ability to compute extinction cross-section
of a scatterer in the long-wavelength limit is provided in Figure \ref{fig:verif_sphere}.
The computed extinction cross-section of a silver nanoparticle in a range of frequencies 
was within 1\% of the analytical value, with the numerical parameters chosen.


%\subsection{LSPR response to BSA}

Figure \ref{fig:2pz_response} shows a red shift of the plasmon resonance frequency peak in presence of the BSA protein.
This result agrees with experimental observations
\cite{TangETal2010, RaphaelETal2013}. Moreover, we observe a decrement of 
the peak which is also present in the results of Tang, et al. \cite{TangETal2010}.
Both results, the red-shift and the decrement of the peak in the presence of 
the proteins, indicate that our boundary element method approach using electrostatic
approximation, is capable of reproducing the characteristic resonance frequency 
shift in LSPR biosensors.

The effect of placing the proteins in a direction that is not aligned with the incoming electric field
is presented by Figure \ref{fig:2pxy_response}. In both cases we can observe that the shift is negligible, 
as the shift is smaller than the resolution between wavelengths ($< 0.25 nm$), resulting in an
effectively zero-shift. These findings are consistent with 
the fact that we have a z-polarized electric field. Having a z-polarized electric
field implies that the cloud of free electrons will osicillate along the z-axis, while 
not in the x and y direction as you can observe in Figure \ref{fig:lspr}. The
effect of the analyte in the latter locations results insignificant since the 
analytes have pratically no interference with the free oscillating electrons. 

In Figure \ref{fig:dist_response} we can observe how the distance between the sensor 
and the analyte affects the shift in resonance frequency. As expected, the shift decays 
as the BSA moves away from the sensor, to the point that is the BSA proteins are placed
$d=2 nm$ away, the the shift is only $0.25 \, nm$. This result shows the potential of \pygbe 
and the electrostatic approach to study sensitivity versus distance.
One has to consider that detection of such
small number of analytes is not possible in experiments, however, these simulations can shine light on
potential improvements that would enhance sensitivity of LSPR biosensors, for example, by using
smaller ligands. 



