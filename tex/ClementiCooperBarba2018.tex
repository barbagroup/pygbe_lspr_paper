
 \documentclass[pre,twocolumn]{revtex4} 

\usepackage{amsfonts}
\usepackage{amsmath}
\usepackage{amssymb}
\usepackage{booktabs}
\usepackage{caption}
\usepackage{color}
\usepackage{comment}
\usepackage{float}
\usepackage{graphicx}
\usepackage{hyperref}
\usepackage[utf8]{inputenc} % allows using accents directly in text, like ÔøΩ
\usepackage{subfig}
\usepackage{xspace}
\usepackage{cuted}

\captionsetup{justification=raggedright,
singlelinecheck=false
}

\newcommand{\pygbe}{\texttt{PyGBe}\xspace}
\newcommand{\gb}{{\small G\,B1\,D4$^\prime$}\xspace}
\newcommand{\gmres}{\textsc{gmres}\xspace}
\newcommand{\bem}{\textsc{bem}\xspace}
\newcommand{\ses}{\textsc{ses}\xspace}
\newcommand{\sam}{\textsc{sam}}
\newcommand{\gpu}{\textsc{gpu}}
\newcommand{\cpu}{\textsc{cpu}}
\newcommand{\apbs}{\textsc{apbs}\xspace}
\newcommand{\nvidia}{\textsc{nvidia}\xspace}
\newcommand{\msms}{\texttt{\textsc{msms}}\xspace}
\newcommand{\amber}{\texttt{\textsc{amber}}\xspace}
\newcommand{\ccby}{\textsc{cc-by}\xspace}
\newcommand{\bigO}{\mathcal{O}}
\renewcommand{\O}[1]{\mathcal{O}(#1)}

\graphicspath{{figs/}} %  PATH to figure files-- change to ./ for submission



\begin{document}


\title{PyGBe-LSPR---Computational nanoplasmonics for biosensing applications}

\author{Natalia C. Clementi}
\email{ncclementi@gwu.edu}
\affiliation{Department of Mechanical \& Aerospace Engineering, The George Washington University, Washington, D.C.}

\author{Christopher D. Cooper}
\email{christopher.cooper@usm.cl}
\affiliation{Department of Mechanical Engineering and Centro Cient\'ifico Tecnol\'ogico de Valpara\'iso, Universidad T\'ecnica Federico Santa Mar\'ia, Valpara\'iso, Chile.}

\author{Lorena A.~Barba}
\email{labarba@gwu.edu}
\affiliation{Department of Mechanical \& Aerospace Engineering, The George Washington University, Washington, D.C.}
%\date{\today}


\begin{abstract} % in revtex4, the abstract must come before the \maketitle command

The phenomenon of localized surface plasmon resonance provides high sensitivity in detecting biomolecules through shifts in resonance frequency when a target is present. 
Computational studies in this field have used the full Maxwell equations with simplified models of a sensor-analyte system, or neglected the analyte altogether. 
In the long-wavelength limit, one can simplify the theory via an electrostatics approximation, while adding geometrical detail in the sensor and analytes (at moderate computational cost).
This work uses the latter approach, expanding the open-source \pygbe code to compute the extinction cross-section of metallic nanoparticles in the presence of any target for sensing.
The target molecule is represented by a surface mesh, based on its crystal structure. 
\pygbe is research software for continuum electrostatics, written in Python with computationally expensive parts accelerated on GPU hardware, via PyCUDA.
It is also accelerated algorithmically via a treecode that offers $\mathcal{O}(N \log N)$ computational complexity. 
These features allow \pygbe to handle problems with half a million boundary elements or more.
In this work, we demonstrate the suitability of \pygbe, extended to compute LSPR response in the electrostatic limit, for biosensing applications. 
Using a model problem consisting of an isolated silver nanosphere in an electric field, our results show grid convergence as $1/N$, and accurate computation of the extinction cross-section as a function of wavelength (compared with an analytical solution).
For a model of a sensor-analyte system, consisting of a spherical silver nanoparticle and a set of bovine serum albumin (BSA) proteins, our results again obtain grid convergence as $1/N$ (with respect to the Richardson extrapolated value).
Computing the LSPR response as a function of wavelength in the presence of BSA proteins captures a red-shift of 0.5 nm in the resonance frequency due to the presence of the analytes at 1-nm distance.
The final result is a sensitivity study of the biosensor model, obtaining the shift in resonance frequency for various distances between the proteins and the nanoparticle.
All results in this paper are fully reproducible, and we have deposited in archival data repositories all the materials needed to run the computations again and re-create the figures. \pygbe is open source under a permissive license and openly developed. Documentation is available at \url{http://barbagroup.github.io/pygbe/docs/}. 
\end{abstract}

\maketitle

% Body of paper.

\section{Introduction} \label{sec:intro}
%* what is known
Localized surface plasmon resonance (LSPR) is an optical effect where an 
electromagnetic wave excites the free electrons on the surface of a metallic nanoparticle.
These vibrations of the electron cloud are known as plasmons, and in LSPR they resonate with the incoming
field (see Figure \ref{fig:lspr}). When this happens, most of the incoming energy
is either absorbed by the nanoparticle, or scattered in different directions,
generating a large shadow behind the scatterer (extinction). In the case of LSPR,
the wavelength of the incoming wave is usually considered to be much larger than 
the size of the nanoparticle.

This principle is used to design 
biosensors, as the resonance frequency is highly dependent of the dielectric environment 
around the scatterer. 
Then, the resonance frequency shifts when an analyte binds to the nanoparticle, 
resulting in a very sensitive sensor \cite{HaesETal2004, HaesVanduyne2002}.

A number of researchers have used numerical simulations to study LSPR \cite{SolisTaboadaObelleiroLiz-MaarzanGarciadeabajo2014}. These mostly rely on the 
solution of Maxwell's equations in some form, solved by means of finite difference time-domain (FDTD),
boundary element, or finite element methods. 
These simulations have been used to study the 
optical properties of dielectric or metallic nanoparticles \cite{Hohenester2018,HohenesterTrugler2012,
JungPedersenSondergaardPedersenLarsenNielsen2010, VideenSun2003,
MayergoyzFredkinZhang2005, MayergoyzZhang2007}, interactions between nanoparticles
and electron beams \cite{GarciadeabajoAizpurua1997, GarciadeabajoHowie2002},
and surface plasmon resonance sensors.
Among the latter application we find biosensors, where researchers have modelled the 
interaction between the metallic nanoparticle and target molecules with highly 
simplified methods \cite{JungCampbellChinowskyMarYee1998,HaesVanduyne2002,DavisGomezVernon2010,AntosiewiczApellClaudioKall2011}.


{\color{red} Not sure how to talk about the different methods, any idea? should
we talk about this? 

Methods: fem (comsol), fdtd (davis and coworker), bem (garcia de abajo and 
howie, bem ++, matlab guy). van duyne and haes? }
{\color{blue} wrote something there, let me know what you think. My impression is that
we shouldnt talk too much on this. We do cite everybody in the 'applications' part of the paragraph, so I dont think we nee to cite them also in the 'methods' part}

\begin{figure}[h] %  figure placement: here, top, bottom, or page
   \centering
   \includegraphics[width=0.35\textwidth]{lspr.pdf} 
   \caption{Localized Surface Plasmon Resonance (LSPR) scheme. }
   \label{fig:lspr}
\end{figure}

%* What is unkown, limitations and gaps

In the biosensor community most progress is achieved 
through experimental analysis, with trial and error procedures {\color{blue} I also have this impression, but do we have a reference to support it? If not, I'd rather avoid it}. 
Using software to assist the design process can play a key role for the manufacture and optimization
of biosensors, as we have access to details that are not available in experimental tests.
For example, empirical studies suggest that the sensitivity of the sensor
is highly dependent on the distance between the nanoparticle and the analyte \cite{HaesETal2004};
these results were modeled with a simplified discrete dipole approximation (DDA), however, 
the analyte was nos considered in the analysis. Moreover, experiments show that 
LSPR sensors are sensitive enough to detect conformational changes of the analytes \cite{HallETal2011}, 
but current simplified models of LSPR are not able to consider such details.

%* Fill the gap

Even though LSPR is an optical effect, electrostatics 
makes a good approximation in the long-wavelength limit. In this work we use
the boundary integral electrostatics solver \pygbe \cite{CooperETal2016} 
to compute the extinction cross-section of metallic nanoparticles, and study how LSPR 
response changes in the presence
of a biomolecule. \pygbe was recently extended to account for complex dielectric constants 
\cite{ClementiETal2017} aiming towards the LSPR biosensing application. We treat Maxwell's
equations quasi-statically \cite{MayergoyzZhang2007} and
use an accurate continuum representation of the biomolecule obtained from the
crystal structure. 

The boundary element solver in \pygbe
is accelerated with a treecode (an $\mathcal{O}(N\log N)$ fast solver), and runs on
graphic processing units (GPUs). Also, the software
\footnote{\url{https://github.com/barbagroup/pygbe}} is shared under the 
BSD 3-clause license and the development repository is available on Github.

{\color{red}  Keeping this structure here for reference until we polish better
the introduction.

What is known:
\begin{itemize}
\item Plasmonic simulations, what applications they cover, etc. (cite Matlab guy, also Garcia de Abajo, Jung+Pedersen, etc. Maybe we can even cite COMSOL here)
\item LSPR: how does it work, what simulations are there in the litarature. Talk about work from, for example, Davis or van Duyne. See page 50 of my thesis (end chapter 2).
\end{itemize}

What is unkown:
\begin{itemize}
\item all developments are trial and error
\item no computational tools to help in the design process
\item example: no understanding of the sensitivity of the system
\item models that consider the nanoparticle and the analyte are extremely simplistic
\end{itemize}

Fill the gap:
\begin{itemize}
\item design of a computational tool that is highly accurate to represent the biomolecule
\end{itemize}
}



%=============
\section{Methods}\label{sec:methods}
\subsection{Far-field scattering} \label{sec:ff_scattering}


\begin{figure}[h] %  figure placement: here, top, bottom, or page
   \centering
   \includegraphics[width=0.45\textwidth]{particle_wave.pdf} 
   \caption{Nanoparticle under electromagnetic wave.}
   \label{fig:part_wave}
\end{figure}

In LSPR computations, we measure the scattered electromagnetic field on a detector
that is located far away from the nanoparticle. When light shines on an object 
like in Figure \ref{fig:part_wave}, the incident wave ($\mathbf{E}_i$,  $\mathbf{B}_i$)
is scattered along the domain, resulting in a total electromagnetic field 
($\mathbf{E}$,  $\mathbf{B}$) that depends on the incoming wave, the particle's 
geometry, and the material constants. In the quasistatic approximation, we only
 need to compute the electric field and the magnetic contribution can be 
neglected \cite{MayergoyzZhang2007}. In the far-field limit, the scattered field
in the outside region ($\Omega_2$) is given by: 

\begin{equation} \label{eq:scat_efield_long_range}
    \mathbf{E}_{2s} = \frac{1}{4\pi\epsilon_2}k^2\frac{e^{ikr}}{r} (\mathbf{\hat{r}} \times \mathbf{p})\times\mathbf{\hat{r}}.
\end{equation} 

where $k=2\pi/\lambda$ is the wave number and $\lambda$ the wavelength, $\mathbf{\hat{r}}$ 
is a unit vector in the direction of the observation point, and $\mathbf{p}$ is
the dipole moment.

We can also obtain the scattered field with the forward-scattering amplitude 
\cite{Jackson}:

\begin{equation} \label{eq:scat_efield_fwa}
    \mathbf{E}_{2s}(\mathbf{r})_{r\to\infty} = \frac{e^{ikr}}{r} \mathbf{F}(\mathbf{k},\mathbf{k}_0),
\end{equation}

where $\mathbf{F}$ is the forward-scattering amplitude, $\mathbf{k}$ is the 
scattered wave vector in the direction of propagation, and $\mathbf{k}_0$ the 
wave vector of the incident field. 

Via these two equations, we use \pygbe to compute the scattered electric field 
and then solve for the forward-scattering amplitude. 

\subsection{Extinction cross-section and Optical theorem} \label{sec:cext_ot}

The extinction cross-section quatifies how much extinction (scattered + absorbed) 
is caused by a particle when this one is shinned with light. It is defined as the
ratio between the extinct energy and the intensity of the incoming wave. 

The optical theorem relates the extinction cross-section with the forward-scattering amplitude. The traditional expression for this relationship applies for non-absorbing media \cite{MayergoyzZhang2007, Jackson}. The expression for absorbing media \cite{BohrenGilra1979, VideenSun2003} was corrected by Mishchenko \cite{Mishchenko2007}, giving the following expression:

\begin{equation} \label{eq:cext_fwa}
    C_\text{ext} = \frac{4\pi}{k^\prime} \operatorname{Im} \left[ \frac{\mathbf{\hat{e}}_i}{|\mathbf{E}_i|}\mathbf{F}(\mathbf{k}=\mathbf{k}_0, \mathbf{k}_0) \right].
\end{equation}


{\color{red}{Chris in Mishenko 2007 paper the equation is not exactly the same 
(look eq 87 in paper), do you have that derivation? How you got to the eq 7.22 
 in your thesis?.}}


Here, $k^\prime$ is the real part of the complex wave number. 

\begin{equation}
    k = k^\prime + ik^{\prime\prime} = \frac{2\pi}{\lambda} n,
\end{equation}

and $n$ is the refraction index of the host medium.


\subsection{Boundary integral formulation} \label{sec:lspr_bem}

From the electrostatic approximation from Mayergoyz and Zhang (2007) 
\cite{MayergoyzZhang2007} we know that the zeroth order term of the magnetic 
field is zero everywhere. Resulting in the following Maxwell equations. 

\begin{align} \label{eq:electrostatic_scatter_E}
\nabla \cdot \mathbf{E}^{(0)}_{1s} &= 0 \qquad \nabla \times \mathbf{E}^{(0)}_{1s} = 0 \nonumber \\
\nabla \cdot \mathbf{E}^{(0)}_{2s} &= 0 \qquad \nabla \times \mathbf{E}^{(0)}_{2s} = 0 \nonumber \\
(\epsilon_1\mathbf{E}^{(0)}_{1s} - \epsilon_2\mathbf{E}^{(0)}_{2s})\cdot\mathbf{n} &= (\epsilon_2-\epsilon_1)\mathbf{E}_i\cdot \mathbf{n}.
\end{align}


\subsubsection{Electrostatic potential under an incoming electric field}


\subsubsection{Boundary integral expression of the dipole moment}

\subsection{Acceleration startegies} \label{sec:acc_strategies}













 

%=============
\section{Results} \label{sec:results}
%!TEX root = ClementiCooperBarba2018.tex

We present results for two kinds of problems. 
The first is a model problem for which an analytical solution is available, 
allowing for a grid-refinement study and code verification using that solution.
It consists of a spherical nanoparticle in a constant electric field, 
where the extinction cross-section can be derived in closed form.
The second set of results use a model for a biosensor detecting a target molecule, 
via frequency shifts in the plasmon resonance of a metallic nanoparticle. 
In this case, since an analytical solution is not available, we can use Richardson 
extrapolation to estimate the errors in a grid-refinement study.
We computed the variation of the extinction cross-section with respect to wavelength
for the isolated nanoparticle, and in the presence of bovine serum albumin (BSA) proteins, 
varying the location of the analytes.
The final result is a sensitivity study of the biosensor model, 
looking at how the peak in frequency response varies with distance of the protein 
to the nanoparticle.

All results were obtained on a lab workstation, built from parts.
Hardware specifications are as follows: 
\begin{itemize}
  \item CPU: Intel Core i7-5930K Haswell-E 6-Core 3.5GHz LGA 2011-v3
  \item RAM: G.SKILL Ripjaws 4 series 32GB (4 x 8GB)
  \item GPU: Nvidia Tesla K40c (with 12 GB memory)
\end{itemize}

\subsection{Grid convergence and verification with an isolated silver nanoparticle} \label{sec:verification}

\noindent In the long-wavelength limit, the electrostatic approximation applies and
the electromagnetic scattering of a small spherical particle can be modeled
by a sphere in a constant electric field. 
Figure \ref{fig:np_elec_field} illustrates this scenario.

%
\begin{figure}[h] %  figure placement: here, top, bottom, or page
   \centering
   \includegraphics[width=0.3\textwidth]{sphere_field_8nm.pdf} 
   \caption{Spherical nanoparticle in a constant electric field.}
   \label{fig:np_elec_field}
\end{figure}
%

This model problem has an analytical solution, which allows us to compare with
the numerical calculations of the extinction cross-section obtained with \pygbe,
for code verification and grid-convergence analysis.
Mishchenko \cite{Mishchenko2007} derived the following analytical result, 
valid for lossy mediums:
%
\begin{equation} 
    C_\text{ext} = \frac{4\pi a^3}{k^\prime} \operatorname{Im}\left(k^2 
                    \frac{\epsilon_p/\epsilon_m -1}{\epsilon_p/\epsilon_m +2}\right).
    \label{eq:an_sol}
\end{equation}
%
Here, $a$ is the radius of the sphere, $k$ is the complex wave number ($k=k^\prime +i k^{\prime\prime}$), $\epsilon_p$ 
is the dielectric constant of the particle, and $\epsilon_m$ is the dielectric constant
of the host medium. If the medium is not lossy, then $k^{\prime\prime}=0$ and $k=k^\prime$.

We completed a grid-convergence study of \pygbe for the extinction
cross-section of a spherical silver nanoparticle of radius 8 nm immersed in water,
under a $z$-polarized electric field with a wavelength of 380 nm and intensity of 
$-0.0037 e/({\AA}^2 \, \epsilon_0)$. In these conditions, water has a dielectric
constant of $1.7972 \, + \, 8.5048^{-09}i$ \cite{JohnsonChristy1972} and silver of
$-3.3877 \, + \, 0.1922i$ \cite{HaleQuerry1972}. 
Table \ref{table:quadparams1} lists the Gauss quadrature points used for each type of boundary element. 
The threshold parameter defining the near-singular region was 0.5 
(refer to the \pygbe documentation, under ``Parameter file format'').
Table \ref{table:treeparams1} shows the treecode and solver parameters for this grid-convergence study.

\begin{table}[h]
    \centering
    \caption{\label{table:quadparams1} Grid-convergence study: Gauss quadrature points; 
    $K$ and $K_{fine}$ are per element; $Nk $ is per element edge (semi-analytical integration). } 
    \begin{tabular}{l l}
    \hline%\toprule
     distant elements: & $K=4$ \\
     near-singular integrals:   & $ K_{fine}=37$ \\
     singular elements:  & $Nk =9$ \\
    \hline%\bottomrule
    \end{tabular}
\end{table}


\begin{table}[h]
    \centering
    \caption{\label{table:treeparams1} Grid-convergence study: treecode and solver parameters.} 
    \begin{tabular}{l l}
    \hline%\toprule
    treecode order of expansion: & $P=15$\\
    MAC                                         & $\theta=0.5$\\
    GMRES tolerance                    & $10^{-5}$\\
    \hline%\bottomrule
    \end{tabular}
\end{table}

The results are shown in Figure \ref{fig:error_sphere_Ag}, where the mesh sizes are
512, 2048, 8192, and 32768 elements. 
The analytical solution with equation \eqref{eq:an_sol} is $C_{ext} = 1854.48$ nm$^2$, 
and the computed errors are as shown in Table \ref{table:err_iso_sphere}.
The observed order of convergence is $0.98$, and the $1/N$ slope in Figure \ref{fig:error_sphere_Ag}
is an indication that the meshes are correctly resolving the numerical solutions with \pygbe. 


\begin{figure}[h] %  figure placement: here, top, bottom, or page
   \centering
   \includegraphics[width=0.45\textwidth]{convergence_sph_Ag_R8_w=380.pdf} 
   \caption{Grid-convergence study for the extinction cross-section of a spherical silver
            nanoparticle, computed with \pygbe.}
   \label{fig:error_sphere_Ag}
\end{figure}



\begin{table}[h]
    \centering
    \caption{\label{table:err_iso_sphere} Percentage error in the grid-convergence cases with an isolated silver nanosphere.} 
    \begin{tabular}{c c}
    \hline%\toprule
    N & \% error \\
    \hline%\midrule
     $512$ & $29.86$ \\
     $2048$ & $7.33$ \\
     $8192$ & $1.9$ \\
     $32768$ & $0.52$ \\
    \hline%\bottomrule
    \end{tabular}
\end{table}

As another verification test of \pygbe, we computed the extinction cross-section of an 
isolated sphere for a range of wavelengths. 
The results are shown in Figure \ref{fig:verif_sphere}, comparing with the analytical solution. 
The values of dielectric constants for each wavelength were obtained by interpolation of 
experimental data \cite{JohnsonChristy1972, HaleQuerry1972}.
For reproducibility of these results, we provide a Jupyter notebook with the code used for this interpolation step.
See section \ref{sec:repro} for details.
We used a mesh with $N=32768$, and relaxed some parameters compared with the grid convergence results shown previously, still yielding errors below $1\%$ at all frequencies.
The parameters used are shown in Tables \ref{table:quadparams2} and \ref{table:treeparams2}.



\begin{table}[h]
    \centering
    \caption{\label{table:quadparams2} Verification: Gauss quadrature points; 
    $K$ and $K_{fine}$ are per element; $Nk $ is per element edge (semi-analytical integration). } 
    \begin{tabular}{l l}
    \hline%\toprule
     distant elements: & $K=4$ \\
     near-singular integrals:   & $ K_{fine}=19$ \\
     singular elements:  & $Nk =9$ \\
    \hline%\bottomrule
    \end{tabular}
\end{table}


\begin{table}[h]
    \centering
    \caption{\label{table:treeparams2} Verification: treecode and solver parameters.} 
    \begin{tabular}{l l}
    \hline%\toprule
    treecode order of expansion: & $P=6$\\
    MAC                                         & $\theta=0.5$\\
    GMRES tolerance                    & $10^{-3}$\\
    \hline%\bottomrule
    \end{tabular}
\end{table}


\begin{figure}[h] %  figure placement: here, top, bottom, or page
   \centering
   \includegraphics[width=0.45\textwidth]{silver_NP_verification.pdf} 
   \caption{Extinction cross-section as a function of wavelength for an $8$-nm
            silver sphere immersed in water.}
   \label{fig:verif_sphere}
\end{figure}

Figure \ref{fig:verif_sphere} shows good agreement between the computed and analytical results, 
evidence that \pygbe can accurately represent the mathematical model. The 
peak in the values of extinction cross-section corresponds to the plasmon resonance of the metallic
nanoparticle under the incoming electric field.


\subsection{LSPR response to bovine serum albumin (BSA)} \label{sec:lspr_response}

Localized Surface Plasmon Resonance (LSPR) biosensors detect a target molecule by monitoring
frequency shifts in the plasmon resonance of metallic nanoparticles, in presence of an analyte \cite{WilletsVandyune2007}.
In this section, we model the response of LSPR biosensors using the expanded capacity of \pygbe.
We consider a spherical silver nanoparticle, and compute the extinction cross-section placing 
bovine serum albumin (BSA) proteins (PDB code: 4FS5) in different locations.
The BSA surface mesh was generated using the open-source software Nanoshaper \cite{Nanoshaper}. 
Nanoshaper takes as inputs the atomic coordinates and radii, which were 
extracted from a \texttt{pqr} file generated with \texttt{pdb2pqr} \cite{Dolinsky04},
 using the built-in \texttt{amber} force field.
 In support of the reproducibility of our results, we deposited the final meshes in the Zenodo data repository.
See section \ref{sec:repro} for details.

\subsubsection{Grid-convergence study} \label{sec:bsa_convergence}
We performed a grid-convergence 
analysis of the system sketched in Figure \ref{fig:setup_conv}. 
Since we compute the extinction cross-section of the spherical nanoparticle only, we 
set a fixed mesh density for the protein and refined the mesh of the
sphere (meshes of 512, 2048, 8192 and 32768 elements). We found that the protein meshed with two
triangles per ${\AA}^2$ was fine enough for the convergence analysis, resulting in $N_{prot} = 98116$ elements. 


\begin{figure}[h] %  figure placement: here, top, bottom, or page
   \centering
   \includegraphics[width=0.15\textwidth]{protein_sphere_sketch.pdf} 
   \caption{Setup for convergence analysis of the LSPR response calculations.}
   \label{fig:setup_conv}
\end{figure}

We used the same physical conditions as in the grid convergence with an isolated silver nanoparticle, and the same numerical parameters, presented in Tables \ref{table:quadparams1} and \ref{table:treeparams1}.
For the protein dielectric constant, we used $2.7514 + 0.2860i$, obtained from the 
functional relationship provided by Phan, et al.~\cite{PhanETal2013}.
The distance between the sensor and the analyte was $d=1$ nm, 
and the BSA protein was oriented such that its dipole moment was aligned with the $y$-axis. 
To obtain the error estimates shown in Figure \ref{fig:error_sphere-bsa} and Table \ref{table:err_bsa_sensor},
we used the Richardson extrapolated value of extinction cross-section as a reference, 
$C_{ext}= 1778.73$ nm$^2$.


\begin{figure}[h] %  figure placement: here, top, bottom, or page
   \centering
   \includegraphics[width=0.45\textwidth]{convergence_bsa_sensor_R8_d=1_w=380.pdf} 
   \caption{Grid-convergence study of extinction cross-section of a spherical silver
            nanoparticle with a BSA protein at $d=1$ nm.}
   \label{fig:error_sphere-bsa}
\end{figure}

The observed order of convergence is $0.99$, and 
Figure \ref{fig:error_sphere-bsa} shows that the error decays with the number
of boundary elements ($1/N$), which is consistent with our verification 
results in Section \ref{sec:verification}. This provides evidence that the
numerical solutions computed with \pygbe are correctly resolved by the meshes.
The percentage errors for the different meshes are presented in Table \ref{table:err_bsa_sensor}.

\begin{table}[h]
    \centering
    \caption{\label{table:err_bsa_sensor} Percentage error of the BSA-sensor system (Fig.~\ref{fig:setup_conv}), with respect to the extrapolated value.} 
    \begin{tabular}{c c}
    \hline%\toprule
    N & \% error \\
    \hline%\midrule
     $512$ & $29.39$ \\
     $2048$ & $7.13$ \\
     $8192$ & $1.82$ \\
     $32768$ & $0.46$ \\
    \hline%\bottomrule
    \end{tabular}
\end{table}

\subsubsection{Resonance frequency shift} \label{sec:bsa_shift}

We computed the LSPR response as a function of the wavelength in the presence 
of the BSA protein. To optimize run-times without compromising accuracy, we used a relaxed
set of parameters, where the protein mesh density was one element per
${\AA}^2$ ($N_{prot}=45140$) and the sphere mesh had $N_{sensor}=32768$ elements. 
These calculations used the same parameters as shown in Tables \ref{table:quadparams2} and \ref{table:treeparams2}.
This parameter choice resulted in a percentage error below 1\%, with respect to the Richardson-extrapolated value.
The run time for each one of these cases was approximately $7.5$ min using one NVIDIA Tesla K40c GPU. 
When two proteins are present, the run time per case is approximately  $15$ min. 

Figure \ref{fig:display_z} shows a visualization of the meshes for these calculations, 
with two proteins placed at a distance $d=1$ nm away from a spherical silver nanoparticle, along the $z$-axis.
The position of the BSA molecule in the $+z$ axis was the same as in the convergence analysis in 
Section \ref{sec:bsa_convergence}, whereas the BSA in the $-z$ position is a 180$^\circ$ 
solid rotation of the BSA in $+z$ about the $y$-axis.
We performed calculations for wavelengths between $382$ nm and $387$ nm, every $0.25$ nm,
which are around the peak seen in Figure \ref{fig:verif_sphere}.

Figure \ref{fig:2pz_response} shows the variation of the extinction cross-section
with respect to wavelength for the isolated nanoparticle ($d=\infty$) and with
BSA proteins placed $d=1$ nm away. 
The result shows a red-shift ($0.5$ nm) in the resonance frequency due to the
presence of the BSA analytes.


\begin{figure}[h] %  figure placement: here, top, bottom, or page
   \centering
    %since we have dots in the names we need to enclose what is before the 
    %extension in { }
   \includegraphics[width=0.45\textwidth]{{2pz_00_ef-0.0037_R8nm}.pdf} 
   \caption{Extinction cross-section as a function of wavelength for a $8$ nm
            silver sphere immersed in water with two BSA proteins placed at
            $\pm 1$ nm away from the surface in the $z$-direction, and at
            infinity (no protein).}
   \label{fig:2pz_response}
\end{figure}


To study the effect of location of the analytes, we re-computed the result placing the 
BSA proteins along the $x$- and $y$-axis, at $\pm 1$ nm, as shown in Figure \ref{fig:display_xy}.
Figure \ref{fig:2pxy_response} shows the results, in each case. 

\begin{figure}[!h] %  figure placement: here, top, bottom, or page
   \centering
   \subfloat{\includegraphics[width=0.45\textwidth]{{2px_00_ef-0.0037_R8nm}.pdf}}
   \quad 
   \subfloat{\includegraphics[width=0.45\textwidth]{{2py_00_ef-0.0037_R8nm}.pdf}} 
   \caption{Extinction cross-section as a function of wavelength for a $8 \, nm$
            silver sphere immersed in water with two BSA proteins placed at
            $\pm 1 \; nm$ away from the surface in the x-direction (top) and
             y-direction (bottom), and at infinity (no protein).}
   \label{fig:2pxy_response}
\end{figure}

\subsubsection{Sensitivity calculations} \label{sec:bsa_sensitivity}

The sensitivity of an LSPR biosensor corresponds to the relationship between the size 
of the resonance frequency shift and the number of analytes bound to the sensor (through a ligand).
Experiments show that the distance between the nanoparticle and the analyte 
affects the sensitivity of the sensor, to the point that
targets placed $15$ nm away from the surface are very hard to detect \cite{HaesETal2004}.
This is a critical issue, considering that common ligands (for example, antibodies) are
larger than $15$ nm. Figure \ref{fig:dist_response} 
shows how the peak varies with the distance at which the analytes ($+z$ and $-z$) are placed.  
In particular, we see a shift of $0.25$ nm when $d=2$ nm to $0.75$ nm when the 
analytes are placed at $d=0.5$ nm. The parameters used in this case remain 
the same as the ones used in Figures \ref{fig:2pz_response} and \ref{fig:2pxy_response} .


\begin{figure}[h] %  figure placement: here, top, bottom, or page
   \centering
   \includegraphics[width=0.45\textwidth]{2pz_lspr_response.pdf} 
   \caption{Extinction cross-section as a function of wavelength for a $8 \, nm$
            silver sphere immersed in water with two BSA proteins placed at
            $2, \, 1 \, \text{and} \, 0.5 \, nm$ away from the surface in the 
            z-direction, and at infinity (no protein)}
   \label{fig:dist_response}
\end{figure}

\begin{figure*}[] %  figure placement: here, top, bottom, or page
   \centering
   \includegraphics[width=0.65\textwidth]{2prot_1nm_z_R8nm.pdf} 
   \caption{Sensor protein display: BSA located at $\pm 1 \, nm$ of the 
            nanoparticle in the z-direction}
   \label{fig:display_z}
\end{figure*}

\begin{figure*}[]

   \centering
   \subfloat{\includegraphics[width=0.65\textwidth]{2prot_1nm_x_R8nm.pdf}} 
%   \caption{Sensor protein display: BSA located at 1 nm of the nanoparticle in the
%            x-direction}
   %\label{fig:display_x}
    \vfill
   \subfloat{\includegraphics[width=0.65\textwidth]{2prot_1nm_y_R8nm.pdf}} 
%   \caption{Sensor protein display: BSA located at 1 nm of the nanoparticle in the
%            y-direction}
    \caption{Sensor protein display: BSA located at $\pm 1 \, nm$ of the nanoparticle in the
            x-direction (top) and y-direction (bottom)}
    \label{fig:display_xy}
\end{figure*}

\subsection{Reproducibility and data management} \label{sec:repro}

To facilitate the reproducibility and replication of our results, 
we consistently release our research code and data with every publication. \pygbe is openly developed and 
shared under the BSD3-clause license via its repository at \url{https://github.com/barbagroup/pygbe}.
We also release all of the data and scripts needed to run the calculations reported in this work, 
as well as the post-processing scripts to reproduce the figures in this paper. 
All the input files necessary to reproduce the computations are available in one Zenodo data set \cite{ClementiETal2018a}. 
Each problem corresponds to a folder, wherein the user can find geometry files (surface meshes), 
configuration files, parameter files, and when it applies, the protein charges (.pqr).
Consistent with our open-source policy, we release a \textit{reproducibility package} for each of the results presented in this paper. 
{\color{red} explain location and which figure correspond to each package after finished repropack}



%=============
\section{Discussion} \label{sec:discussion}
%!TEX root = ClementiCooperBarba2018.tex

Extending \pygbe to the LSPR biosensing application required considerable code 
modifications and added functionality. The results presented in the previous section 
offer evidence to build confidence on the suitability of the mathematical model 
and the correctness of the code.
The grid-convergence study with a nanosphere under a constant electric field 
shows a $1/N$ rate of convergence, consistent with convergence results 
in previous work using \pygbe \cite{CooperBardhanBarba2013}.
Further verification of \pygbe's new ability to compute extinction cross-section
of a scatterer in the long-wavelength limit is provided in Figure \ref{fig:verif_sphere}.
The computed extinction cross-section of a silver nanoparticle in a range of frequencies 
is within 1\% of the analytical value, with the numerical parameters chosen.
This level of accuracy is likely sufficient, given that experimental uncertainty in 
the values of the dielectric constant for silver is in the order of 1\%, also \cite{JohnsonChristy1972}.

%\subsection{LSPR response to BSA}

Figure \ref{fig:2pz_response} shows a red shift of the plasmon resonance frequency peak in presence of the BSA proteins.
Experimental observations of Tang, et al.~\cite{TangETal2010} with silver nanoparticles of approximately 17 nm in diameter and BSA proteins in solution revealed a red shift upon adding the proteins. 
Similar to the effect we see with our model, they observed as well a decrement of 
the peak amplitude.
Moreover, recent experiments \cite{PuETal2018} report a resonance frequency for a silver nanoparticle in the presence of BSA proteins of between 380 and 400 nm, which is consistent with our results.
Other experiments \cite{RaphaelETal2013} also report a red shift in the resonance frequency in the presence of (different) proteins.
Our boundary element method approach using electrostatic
approximation is thus able to capture the characteristic resonance-frequency 
shift of LSPR biosensors.

With the electric field aligned in the $z$-direction, placing the proteins at a distance
in the $x$ or $y$ directions from the nanoparticle shows a negligible shift in the 
resonance peak: the shifts in Figure \ref{fig:2pxy_response} 
are smaller than the resolution between wavelengths ($< 0.25$ nm).
This finding is consistent with the free electrons oscillating in the $z$ direction
under a $z$-polarized electric field, and not in the $x$ and $y$ directions
(see Figure \ref{fig:lspr}). 
The analytes have a marked effect when placed in the $z$ direction, where
they can interfere with the free oscillating electrons. 

Figure \ref{fig:dist_response} shows how the  shift in resonance frequency varies 
with the distance between the sensor and the analyte. As expected, the shift decays 
as the BSA moves away from the sensor, to the point that if the BSA proteins are placed
$d=2$ nm away, the shift is only $0.25$ nm. This result shows the potential of \pygbe 
and the electrostatic approach to study biosensor sensitivity with distance.
Note that possible quantum effects (e.g., tunneling) at $d=0.5$nm are ignored with our classical model.
Even if this distance could be close to or in the quantum regime, evidence that classical theory is valid at this distance in similar systems has been reported 
\cite{SavageETal2012, EstebanETal2012}.  

Even though there is evidence that techniques such as Plasmon Enhanced Raman Scattering are capable of detecting all the way to single molecules \cite{ZhangZhangETal2013}, 
as far as we know, there is no evidence of purely LSPR approaches that can sense such low concentration of analytes.
These computational studies can shine light on  potential improvements that would enhance sensitivity of LSPR biosensors, for example, by using smaller ligands. 

We are not aware of other LSPR simulations where the molecular details of the analyte are considered, however, similar calculations could be performed with other software. 
For example, BEM++ \cite{SmigajETal2015} also models the system as a set of boundary integral equations, discretized in flat triangular panels. 
This software uses the Galerkin approach and algorithmic acceleration via hierarchical matrices, which is slower and less memory efficient than the treecode and limits the accessible problem sizes.
The Matlab toolbox MNPBEM \cite{HohenesterTrugler2012} is another alternative software designed to simulate scattering of metallic nanoparticles.
Its BEM implementation is similar to \pygbe as it uses a centroid collocation scheme on flat triangular panels, but differs in the algorithmic acceleration technique, which is also based on hierarchical matrices rather than a treecode. 
This results in higher memory usage compared to our code, making it harder to simulate large analytes in detail.
Commercial finite-element or finite-difference solvers could also be used in this application, for example, COMSOL. 
These volumetric approaches, however, struggle to correctly impose the zero boundary condition at infinity, which is exactly met for a BEM formulation.


\section{Conclusion}
%\input{conclusion}

In this work, we combined the implicit-solvent model of electrostatics interactions in \pygbe 
with a long-wavelength representation of LSPR response in nanoparticles. 
We extended \pygbe to work with complex-valued quantities, and added functionality to 
include an imposed electric field and compute relevant quantities 
(dipole moment, extinction cross-section). 
Previous work with \pygbe showed its suitability for computing 
biomolecular electrostatics considering solvent-filled cavities and Stern layers \cite{CooperBardhanBarba2013}, 
and for protein-surface electrostatic interactions \cite{CooperBarba2016}.
This latest extension can offer a valuable computational approach to study nanoplasnomics and aid in the design of LSPR biosensors. 
Other research software that could be used in this setting includes  
BEM++ \cite{SmigajETal2015} and a Matlab toolbox called MNPBEM \cite{HohenesterTrugler2012}. 
We believe in both cases the size of problems that can be treated, in terms of number of boundary elements, may not be enough to resolve the details of target biomolecules from their crystal structure. 
Thanks to algorithmic acceleration with a treecode, and hardware acceleration with GPUs, \pygbe is able to compute problems with half a million elements, or more.



\begin{acknowledgments}

CDC acknowledges the financial support from CONICYT through projects FONDECYT Iniciaci\'on 11160768 and Basal FB0821.
\end{acknowledgments}

% Create the reference section using BibTeX:
%\bibliographystyle{} % revtex ournal sub-style automatically sets this
\bibliography{compbio,bem,scicomp,fastmethods,biosensors} %don't leave spaces between elements, it throws error

\end{document}
