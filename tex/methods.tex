\subsection{Far-field scattering} \label{sec:ff_scattering}


\begin{figure}[h] %  figure placement: here, top, bottom, or page
   \centering
   \includegraphics[width=0.45\textwidth]{particle_wave.pdf} 
   \caption{Nanoparticle under electromagnetic wave.}
   \label{fig:part_wave}
\end{figure}

In LSPR computations, we measure the scattered electromagnetic field on a detector
that is located far away from the nanoparticle. When light shines on an object 
like in Figure \ref{fig:part_wave}, the incident wave ($\mathbf{E}_i$,  $\mathbf{B}_i$)
is scattered along the domain, resulting in a total electromagnetic field 
($\mathbf{E}$,  $\mathbf{B}$) that depends on the incoming wave, the particle's 
geometry, and the material constants. In the quasistatic approximation, we only
 need to compute the electric field and the magnetic contribution can be 
neglected \cite{MayergoyzZhang2007}. In the far-field limit, the scattered field
in the outside region ($\Omega_2$) is given by: 

\begin{equation} \label{eq:scat_efield_long_range}
    \mathbf{E}_{2s} = \frac{1}{4\pi\epsilon_2}k^2\frac{e^{ikr}}{r} (\mathbf{\hat{r}} \times \mathbf{p})\times\mathbf{\hat{r}}.
\end{equation} 

\noindent where $k=2\pi/\lambda$ is the wave number and $\lambda$ the wavelength, $\mathbf{\hat{r}}$ 
is a unit vector in the direction of the observation point, and $\mathbf{p}$ is
the dipole moment.

We can also obtain the scattered field with the forward-scattering amplitude 
\cite{Jackson}:

\begin{equation} \label{eq:scat_efield_fwa}
    \mathbf{E}_{2s}(\mathbf{r})_{r\to\infty} = \frac{e^{ikr}}{r} \mathbf{F}(\mathbf{k},\mathbf{k}_0),
\end{equation}

\noindent where $\mathbf{F}$ is the forward-scattering amplitude, $\mathbf{k}$ is the 
scattered wave vector in the direction of propagation, and $\mathbf{k}_0$ the 
wave vector of the incident field. 

Via these two equations, we use \pygbe to compute the scattered electric field 
and then solve for the forward-scattering amplitude. 

\subsection{Extinction cross-section and Optical theorem} \label{sec:cext_ot}

The extinction cross-section quatifies how much extinction (scattered + absorbed) 
is caused by a particle when this one is shinned with light. It is defined as the
ratio between the extinct energy and the intensity of the incoming wave. 

The optical theorem relates the extinction cross-section with the forward-scattering amplitude. The traditional expression for this relationship applies for non-absorbing media \cite{MayergoyzZhang2007, Jackson}. The expression for absorbing media \cite{BohrenGilra1979, VideenSun2003} was corrected by Mishchenko \cite{Mishchenko2007}, giving the following expression:

\begin{equation} \label{eq:cext_fwa}
    C_\text{ext} = \frac{4\pi}{k^\prime} \operatorname{Im} \left[ \frac{\mathbf{\hat{e}}_i}{|\mathbf{E}_i|}\mathbf{F}(\mathbf{k}=\mathbf{k}_0, \mathbf{k}_0) \right].
\end{equation}


{\color{red}{Chris in Mishenko 2007 paper the equation is not exactly the same 
(look eq 87 in paper), do you have that derivation? How you got to the eq 7.22 
 in your thesis?.}}


Here, $k^\prime$ is the real part of the complex wave number. 

\begin{equation}
    k = k^\prime + ik^{\prime\prime} = \frac{2\pi}{\lambda} n,
\end{equation}

and $n$ is the refraction index of the host medium.


\subsection{Boundary integral formulation} \label{sec:lspr_bem}

\subsubsection{Electrostatic potential under an incoming electric field}

From the electrostatic approximation from Mayergoyz and Zhang (2007) 
\cite{MayergoyzZhang2007} we know that the zeroth order term of the magnetic 
field is zero everywhere. Resulting in the following Maxwell equations. 

\begin{align} \label{eq:electrostatic_scatter_E}
\nabla \cdot \mathbf{E}^{(0)}_{1s} &= 0 \qquad \nabla \times \mathbf{E}^{(0)}_{1s} = 0 \nonumber \\
\nabla \cdot \mathbf{E}^{(0)}_{2s} &= 0 \qquad \nabla \times \mathbf{E}^{(0)}_{2s} = 0 \nonumber \\
(\epsilon_1\mathbf{E}^{(0)}_{1s} - \epsilon_2\mathbf{E}^{(0)}_{2s})\cdot\mathbf{n} &= (\epsilon_2-\epsilon_1)\mathbf{E}_i\cdot \mathbf{n}.
\end{align}

Where the subscript $s$ ($i$) stands for scattered (incident) field. The curl of
$\mathbf{E}^{(0)}_{js}$ for $j=1,2$ is zero, therefore, there is a scalar potential
such that $-\nabla \phi_js = \mathbf{E}^{(0)}_js$. Using this, we can rewrite 
equation \eqref{eq:electrostatic_scatter_E} as:

\begin{align} \label{eq:electrostatic_scatter}
\nabla^2 \phi_{1s} &= 0 \qquad \nabla^2 \phi_{2s} = 0 \qquad\text{on $\Omega_1$, $\Omega_2$} \nonumber \\
\epsilon_1\frac{\partial\phi_{1s}}{\partial \mathbf{n}} - \epsilon_2\frac{\partial\phi_{2s}}{\partial\mathbf{n}} &= (\epsilon_2-\epsilon_1)\frac{\partial\phi_i}{\partial\mathbf{n}} \quad \phi_{1s} = \phi_{2s} \quad \text{on $\Gamma$}.
\end{align}

\paragraph{Week formulation and layer operators}


The weak formulation of Laplace equation with test function $w$:

\begin{equation} \label{eq:lap_weak}
\int_\Omega \nabla^2 \phi(\mathbf{r}_\Omega') w(\mathbf{r}_\Omega') \text{d} \Omega^\prime= 0.
\end{equation}

\noindent where the evaluation point is $\mathbf{r}_\Omega$ a location in the domain $\Omega$.

If we use the Laplace's free-space Green's function as the test function $w$ we
get:

\begin{equation} \label{eq:lap_weak2}
\int_\Omega \nabla^2 \phi(\mathbf{r}'_\Omega) G_L(\mathbf{r}_\Omega,\mathbf{r}'_\Omega) \text{d} \Omega^\prime= 0.
\end{equation}

Manipulating the integrand using the product rule and later the divergence 
theorem, we get:

\begin{equation} \label{eq:lap_bie_dom}
\phi(\mathbf{r}_\Omega) = \int_\Gamma G_L(\mathbf{r}_\Omega,\mathbf{r}'_\Gamma)  \frac{\partial} {\partial \mathbf{n}} \phi(\mathbf{r}'_\Gamma)  \text{d} \Gamma^\prime - \int_\Gamma \phi(\mathbf{r}'_\Gamma)  \frac{\partial}{\partial \mathbf{n}} G_L(\mathbf{r}_\Omega,\mathbf{r}'_\Gamma) \text{d} \Gamma^\prime
\end{equation}

\noindent where \eqref{eq:lap_bie_dom}, $\mathbf{r}$ can be anywhere in the domain $\Omega$, 
and $\mathbf{r}'$ runs only on the boundary $\Gamma$. This equation has a 
singularity when $\mathbf{r}=\mathbf{r}'$. To handle thisproblem, we perform the
integral on a surface $\Gamma'$ that is like $\Gamma$ but with a hemisphere of 
radius $\varepsilon$ center at $\mathbf{r}$. We split the integrals into the part
that we have no singularity and the part that has the hemisphere. After solving these
equations when $\varepsilon \to 0$, equation \eqref{eq:lap_bie_dom} results in:

\begin{equation} \label{eq:lap_bie}
\frac{\phi(\mathbf{r}_\Gamma)}{2} +  \int_\Gamma \phi(\mathbf{r}'_\Gamma)  \frac{\partial}{\partial \mathbf{n}} G_L(\mathbf{r}_\Gamma,\mathbf{r}'_\Gamma) \text{d} \Gamma^\prime = \int_\Gamma G_L(\mathbf{r}_\Gamma,\mathbf{r}'_\Gamma)  \frac{\partial} {\partial \mathbf{n}} \phi(\mathbf{r}'_\Gamma)  \text{d} \Gamma^\prime,
\end{equation}

\noindent where these are Cauchy principal value integrals.

Using the single and double layer operators:

{\color{red} Chris, V is single layer operator but is it K the double layer one? 
In your thesis you have that the double layer is W and you refers as K as
"an operator" eq 2.108 and 2.112 in your thesis}

\begin{equation}\label{eq:single_layer}
V^{\mathbf{r}_\Gamma}_L (\psi(\mathbf{r}_\Gamma)) = \int_\Gamma \psi(\mathbf{r}'_\Gamma) G_L(\mathbf{r}_\Gamma, \mathbf{r}'_\Gamma) \text{d} \Gamma.
\end{equation}

\begin{equation}\label{eq:double_layer}
K^{\mathbf{r}_\Gamma}_L (\psi(\mathbf{r}_\Gamma)) = \int_\Gamma \psi(\mathbf{r}'_\Gamma) \frac{\partial}{\partial \mathbf{n}}G_L(\mathbf{r}_\Gamma, \mathbf{r}'_\Gamma) \text{d} \Gamma.
\end{equation}

We can rewrite equation \eqref{eq:lap_bie} using the operator notation, as:

\begin{equation} \label{eq:lap_operator}
\left[ \frac{\mathbb{I}}{2} + K_L^{\mathbf{r}_\Gamma} \right] \left( \phi_\Gamma \right) = V_L^{\mathbf{r}_\Gamma} \left( \frac{\partial}{\partial \mathbf{n}} \phi_\Gamma \right),
\end{equation}

\noindent where $\mathbb{I}$ is the identity operator.

Following the same stepps, we can rewrite the Laplace equations in eqaution
\eqref{eq:electrostatic_scatter} as:

%
\begin{align} \label{eq:integral_eq_lspr_nobc}
\frac{\phi_{1s,\Gamma}}{2}+ K_{L}^{\Gamma}(\phi_{1s,\Gamma}) - V_{L}^{\Gamma} \left(\frac{\partial}{\partial \mathbf{n}}\phi_{1s,\Gamma} \right) = 0&  \nonumber \\
\frac{\phi_{2s,\Gamma}}{2} - K_{L}^{\Gamma}(\phi_{2s,\Gamma}) + V_{L}^{\Gamma} \left( \frac{\partial}{\partial \mathbf{n}} \phi_{2s,\Gamma} \right) = 0& \quad \text{on $\Gamma$,}
\end{align}

Applying the interface conditions of Equation \eqref{eq:electrostatic_scatter},
we get:

\begin{align} \label{eq:integral_eq_lspr}
\frac{\phi_{1s,\Gamma}}{2}+ K_{L}^{\Gamma}(\phi_{1s,\Gamma}) - V_{L}^{\Gamma} \left(\frac{\partial}{\partial \mathbf{n}}\phi_{1s,\Gamma} \right) = 0&  \nonumber \\
\frac{\phi_{1s,\Gamma}}{2} - K_{L}^{\Gamma}(\phi_{1s,\Gamma}) + \frac{1}{\epsilon_2}V_{L}^{\Gamma} \left( \epsilon_1 \frac{\partial}{\partial \mathbf{n}} \phi_{1s,\Gamma} - (\epsilon_2-\epsilon_1) \frac{\partial}{\partial \mathbf{n}} \phi_{i,\Gamma} \right) = 0& \quad \text{on $\Gamma$.}
\end{align}


\paragraph{Discretization and Linear system}


We discretize the surface into flat triangles, and assume that  $\phi$ and 
$\partial \phi/\partial \mathbf{n}$ are constant within each panel. Then, we can
write the layer operators in their discretized form:

\begin{align} \label{eq:layers_disc}
V_{L,\text{disc}}^{\mathbf{r}_i} \left( \frac{\partial}{\partial \mathbf{n}} \phi(\mathbf{r}_{\Gamma}) \right) &= \sum_{j=1}^{N_p} \frac{\partial}{\partial \mathbf{n}} \phi(\mathbf{r}_{\Gamma_j}) \int_{\Gamma_j} G_L(\mathbf{r}_{i},\mathbf{r}_{\Gamma_j})  \mathrm{d} \Gamma_j  \nonumber \\
K_{L,\text{disc}}^{\mathbf{r}_i}(\phi(\mathbf{r}_{\Gamma})) &=  \sum_{j=1}^{N_p}\phi(\mathbf{r}_{\Gamma_j})\int_{\Gamma_j} \frac{\partial}{\partial \mathbf{n}} \left[ G_L(\mathbf{r}_{i},\mathbf{r}_{\Gamma_j}) \right]\mathrm{d} \Gamma_j
\end{align}

\noindent where $N_p$ is the number of discretization elements of $\Gamma$, 
and $\phi(\mathbf{r}_{\Gamma_j})$ and $\frac{\partial}{\partial \mathbf{n}} 
\phi(\mathbf{r}_{\Gamma_j})$ are the values of $\phi$ and 
$\frac{\partial \phi}{\partial \mathbf{n}}$ on panel $\Gamma_j$.


Then we can write equation \eqref{eq:integral_eq_lspr} in matrix form as:
%
 \begin{equation} \label{eq:matrix_lspr}
 \left[
    \begin{matrix} 
       \frac{1}{2} + K_{L}^{\Gamma} & -V_{L}^{\Gamma}  \vspace{0.2cm} \\
       \frac{1}{2} - K_{L}^{\Gamma} &  \frac{\epsilon_1}{\epsilon_2} V_{L}^{\Gamma}  \vspace{0.2cm} 
    \end{matrix}
    \right] \left[ 
    \begin{matrix} 
       \phi_{1s,\Gamma} \vspace{0.2cm} \\
       \frac{\partial}{\partial \mathbf{n}} \phi_{1s,\Gamma} \vspace{0.2cm}
    \end{matrix} 
     \right] =   
    \left[
    \begin{matrix} 
       0 \\
       V_{L}^{\Gamma} \left(\frac{\epsilon_2-\epsilon_1}{\epsilon_2}\right) \frac{\partial\phi_i}{\partial\mathbf{n}} \vspace{0.2cm} 
    \end{matrix}
    \right]
 \end{equation}

\noindent where the elements of the matrices are
%
\begin{align} \label{eq:layers_element}
V_{L,ij}^{\Gamma} &= \int_{\Gamma_j} G_L(\mathbf{r}_{\Gamma_i},\mathbf{r}_{\Gamma_j})  \mathrm{d} \Gamma_j \nonumber \\
K_{L,ij}^{\Gamma} &= \int_{\Gamma_j} \frac{\partial}{\partial \mathbf{n}} \left[ G_L(\mathbf{r}_{\Gamma_i},\mathbf{r}_{\Gamma_j}) \right]\mathrm{d} \Gamma_j
\end{align}

\noindent with $\mathbf{r}_{\Gamma_i}$ at the center of $\Gamma_i$.


\subsubsection{Boundary integral expression of the dipole moment}



\subsection{Acceleration startegies} \label{sec:acc_strategies}













