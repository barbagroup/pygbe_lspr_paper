%!TEX root = ClementiCooperBarba2018.tex

The original implementation of \pygbe used continuum electrostatic theory to compute
the solvation energy of biomolecular systems. In that setting, biomolecules are modeled as 
dielectric cavities inside an infinite continuum solvent, 
leading to a Poisson equation inside the molecules and Laplace or Poisson-Boltzmann in the solvent medium (with appropriate boundary conditions).
This set of partial differential equations can be 
expressed with the corresponding boundary integral equation along the molecular interface, 
which \pygbe solves using a boundary element method \cite{CooperBardhanBarba2013,CooperClementiBarba2015}.

The present work extends \pygbe to the LSPR biosensing application. 
In the long-wavelength limit, Maxwell's equations can be approximated by a Laplace equation,
which permits using the methods implemented in \pygbe, with modifications
to allow for complex-valued permittivities, and to include the
effect of an external electric field.
This section describes the mathematical formulation for computing electromagnetic scattering 
in the long-wavelength setting, and develops the associated boundary integral equations 
and their discretized form.

\subsection{Scattering of small particles} \label{sec:scattering_small}

Electromagnetic scattering is usually modeled with Maxwell's equations.
When the wavelength of the incoming wave is much larger than the
scatterer, these can be reduced to a \emph{quasi-static} 
first-order approximation \cite{MayergoyzZhang2007}:
%
\begin{align} \label{eq:electrostatic_scatter_E}
\nabla \cdot \mathbf{E}_{1s} &= 0 \qquad \nabla \times \mathbf{E}_{1s} = 0, \nonumber \\
\nabla \cdot \mathbf{E}_{2s} &= 0 \qquad \nabla \times \mathbf{E}_{2s} = 0, \nonumber \\
\text{with interface conditions, } \nonumber \\
(\epsilon_1\mathbf{E}_{1s} - \epsilon_2\mathbf{E}_{2s})\cdot\mathbf{n} &= (\epsilon_2-\epsilon_1)\mathbf{E}_i\cdot \mathbf{n}.
\end{align}
%
In Equation \eqref{eq:electrostatic_scatter_E}, $\mathbf{E}_{1s}$ and $\mathbf{E}_{2s}$ 
are the electric fields of the scattered wave in the nanoparticle and host regions, respectively 
(see Figure \ref{fig:part_wave}), 
$\mathbf{E}_{i}$ is the field of the incoming wave, and $\epsilon_1$ 
and $\epsilon_2$ are the permittivities.
This approximation decouples the electric and magnetic fields, neglects the magnetic field, 
and describes the electric field as a curl-free vector field.
Hence, we can reformulate Equation \eqref{eq:electrostatic_scatter_E} with a scalar potential
($-\nabla \phi_{js} = \mathbf{E}_{js}$), as follows:
%
\begin{align} \label{eq:electrostatic_scatter}
\nabla^2 \phi_{1s} &= 0 \qquad \nabla^2 \phi_{2s} = 0 \qquad\text{on $\Omega_1$, $\Omega_2$} \nonumber \\
\epsilon_1\frac{\partial\phi_{1s}}{\partial \mathbf{n}} - \epsilon_2\frac{\partial\phi_{2s}}{\partial\mathbf{n}} &= (\epsilon_2-\epsilon_1)\frac{\partial\phi_i}{\partial\mathbf{n}} \quad \phi_{1s} = \phi_{2s} \quad \text{on $\Gamma$}.
\end{align}
%
Equation \eqref{eq:electrostatic_scatter} is an electrostatic equation 
with an imposed electric field $\mathbf{E}_i=-\nabla\phi_i$, where $\Gamma$ 
is the boundary between regions $\Omega_1$ and $\Omega_2$.

\begin{figure}%[h] %  figure placement: here, top, bottom, or page
   \centering
   \includegraphics[width=0.45\textwidth]{particle_wave.pdf} 
   \caption{Nanoparticle interacting with an electromagnetic wave.}
   \label{fig:part_wave}
\end{figure}

\subsection{Far-field scattering} \label{sec:ff_scattering}

In LSPR, the scattered electromagnetic wave is measured by a detector located far away 
from the scatterer (nanoparticle), and plasmon resonance is identified when the energy 
detected is minimum. In the far-field limit, the scattered field
in the outside region ($\Omega_2$) is given by: 

\begin{equation} \label{eq:scat_efield_long_range}
    \mathbf{E}_{2s} = \frac{1}{4\pi\epsilon_2}k^2\frac{e^{ikr}}{r} (\mathbf{\hat{r}} \times \mathbf{p})\times\mathbf{\hat{r}}.
\end{equation} 

\noindent where $k=2\pi/\lambda$ is the wave number and $\lambda$ the wavelength, $\mathbf{\hat{r}}$ 
is a unit vector in the direction of the observation point, and $\mathbf{p}$ is
the dipole moment.
We can obtain the scattered field using the 
scattering amplitude \cite{Jackson}:

\begin{equation} \label{eq:scat_efield_fwa}
    \mathbf{E}_{2s}(\mathbf{r})_{r\to\infty} = \frac{e^{ikr}}{r} \mathbf{F}(\mathbf{k},\mathbf{k}_0),
\end{equation}

\noindent where $\mathbf{F}$ is the scattering amplitude, $\mathbf{k}$ is the 
scattered wave vector in the direction of propagation, and $\mathbf{k}_0$ the 
wave vector of the incident field. 

\subsection{Extinction cross-section and optical theorem} \label{sec:cext_ot}

The extinction cross-section ($C_\text{ext}$) is a measure of the energy that 
does not reach the detector, either because of scattering in other directions,
or absorption. This quantity is defined as the ratio between the lost energy and 
the intensity of the incoming wave, and has units of area. 
The extinction cross-section peaks at resonance of plasmons.

The extinction cross-section is related to the forward-scattering amplitude via the optical theorem. 
The traditional expression for this relationship applies for non-absorbing media 
\cite{MayergoyzZhang2007, Jackson}; 
Mishchenko \cite{Mishchenko2007} corrected it for absorbing media, 
giving an expression that can be re-written using Jackson's notation \cite{Jackson} as follows:

\begin{equation} \label{eq:cext_fwa}
    C_\text{ext} = \frac{4\pi}{k^\prime} \operatorname{Im} \left[ \frac{\mathbf{\hat{e}}_i}{|\mathbf{E}_i|}\mathbf{F}(\mathbf{k}=\mathbf{k}_0, \mathbf{k}_0) \right].
\end{equation}

\noindent Here, $k^\prime$ is the real part of the complex wave number, 

\begin{equation}
    k = k^\prime + ik^{\prime\prime} = \frac{2\pi}{\lambda} n,
\end{equation}

\noindent and $n$ is the refraction index of the host medium.

Combining Equations \eqref{eq:scat_efield_long_range} and \eqref{eq:scat_efield_fwa},
we can compute the scattering amplitude to then obtain the extinction cross-section 
with Equation \eqref{eq:cext_fwa}


\subsection{The boundary element method} \label{sec:lspr_bem}

\subsubsection{Electrostatic potential of a nanoparticle under an electric field} \label{sec:pot_elec_field}

\paragraph{Integral formulation}

Using Green's second identity, the system of partial differential equations 
in Equation \eqref{eq:electrostatic_scatter} can be rewritten as a system 
of boundary integral equations \cite{BrebbiaDominguez1992}. Evaluating on the surface $\Gamma$, this
becomes
%
\begin{align} \label{eq:integral_eq_lspr_nobc}
\frac{\phi_{1s,\Gamma}}{2}+ K_{L}^{\Gamma}(\phi_{1s,\Gamma}) - V_{L}^{\Gamma} \left(\frac{\partial}{\partial \mathbf{n}}\phi_{1s,\Gamma} \right) = 0&  \nonumber \\
\frac{\phi_{2s,\Gamma}}{2} - K_{L}^{\Gamma}(\phi_{2s,\Gamma}) + V_{L}^{\Gamma} \left( \frac{\partial}{\partial \mathbf{n}} \phi_{2s,\Gamma} \right) = 0&,
\end{align}
%
where $V$ and $K$ are the single- and double-layer operators, respectively:
%
\begin{equation}\label{eq:single_layer}
V^{\Gamma}_L (\psi(\mathbf{r}_\Gamma)) = \oint_\Gamma \psi(\mathbf{r}'_\Gamma) G_L(\mathbf{r}_\Gamma, \mathbf{r}'_\Gamma) \text{d} \Gamma',
\end{equation}
%
\begin{equation}\label{eq:double_layer}
K^{\Gamma}_L (\psi(\mathbf{r}_\Gamma)) = \oint_\Gamma \psi(\mathbf{r}'_\Gamma) \frac{\partial}{\partial \mathbf{n}}G_L(\mathbf{r}_\Gamma, \mathbf{r}'_\Gamma) \text{d} \Gamma'.
\end{equation}
%
\noindent Here, $G_L$ is the free-space Green's function of the Laplace equation:
%
\begin{equation}
G_L(\mathbf{r},\mathbf{r}') = \frac{1}{4\pi|\mathbf{r}-\mathbf{r}'|}
\end{equation}

\noindent Applying the interface conditions of Equation \eqref{eq:electrostatic_scatter},
leads to:
%
\begin{align} \label{eq:integral_eq_lspr}
\frac{\phi_{1s,\Gamma}}{2}+ K_{L}^{\Gamma}(\phi_{1s,\Gamma}) - V_{L}^{\Gamma} \left(\frac{\partial}{\partial \mathbf{n}}\phi_{1s,\Gamma} \right) &= 0  \nonumber \\
\frac{\phi_{1s,\Gamma}}{2} - K_{L}^{\Gamma}(\phi_{1s,\Gamma}) + \frac{\epsilon_1}{\epsilon_2}V_{L}^{\Gamma} \left( \frac{\partial}{\partial \mathbf{n}} \phi_{1s,\Gamma}  \right) &= \nonumber \\
 \frac{\epsilon_2-\epsilon_1}{\epsilon_2}V_{L}^{\Gamma}\left( \frac{\partial}{\partial \mathbf{n}} \phi_{i,\Gamma} \right)\quad \text{on $\Gamma$.}
\end{align}

%The weak formulation of Laplace equation with test function $w$:

%\begin{equation} \label{eq:lap_weak}
%\int_\Omega \nabla^2 \phi(\mathbf{r}_\Omega') w(\mathbf{r}_\Omega') \text{d} \Omega^\prime= 0.
%\end{equation}

%\noindent where the evaluation point is $\mathbf{r}_\Omega$ a location in the domain $\Omega$.

%If we use the Laplace's free-space Green's function as the test function $w$ we
%get:

%\begin{equation} \label{eq:lap_weak2}
%\int_\Omega \nabla^2 \phi(\mathbf{r}'_\Omega) G_L(\mathbf{r}_\Omega,\mathbf{r}'_\Omega) \text{d} \Omega^\prime= 0.
%\end{equation}

%Manipulating the integrand using the product rule and later the divergence 
%theorem, we get:

%\begin{equation} \label{eq:lap_bie_dom}
%\phi(\mathbf{r}_\Omega) = \oint_\Gamma G_L(\mathbf{r}_\Omega,\mathbf{r}'_\Gamma)  \frac{\partial} {\partial \mathbf{n}} \phi(\mathbf{r}'_\Gamma)  \text{d} \Gamma^\prime - \oint_\Gamma \phi(\mathbf{r}'_\Gamma)  \frac{\partial}{\partial \mathbf{n}} G_L(\mathbf{r}_\Omega,\mathbf{r}'_\Gamma) \text{d} \Gamma^\prime
%\end{equation}

%\noindent where \eqref{eq:lap_bie_dom}, $\mathbf{r}$ can be anywhere in the domain $\Omega$, 
%and $\mathbf{r}'$ runs only on the boundary $\Gamma$. This equation has a 
%singularity when $\mathbf{r}=\mathbf{r}'$. To handle this problem, we perform the
%integral on a surface $\Gamma'$ that is like $\Gamma$ but with a hemisphere of 
%radius $\varepsilon$ center at $\mathbf{r}$. We split the integrals into the part
%that we have no singularity and the part that has the hemisphere. After solving these
%equations when $\varepsilon \to 0$, equation \eqref{eq:lap_bie_dom} results in:

%\begin{equation} \label{eq:lap_bie}
%\frac{\phi(\mathbf{r}_\Gamma)}{2} +  \oint_\Gamma \phi(\mathbf{r}'_\Gamma)  \frac{\partial}{\partial \mathbf{n}} G_L(\mathbf{r}_\Gamma,\mathbf{r}'_\Gamma) \text{d} \Gamma^\prime = \oint_\Gamma G_L(\mathbf{r}_\Gamma,\mathbf{r}'_\Gamma)  \frac{\partial} {\partial \mathbf{n}} \phi(\mathbf{r}'_\Gamma)  \text{d} \Gamma^\prime,
%\end{equation}

%\noindent where these are Cauchy principal value integrals.

%Using the single and double layer operators:

%{\color{red} Chris, V is single layer operator but is it K the double layer one? 
%In your thesis you have that the double layer is W and you refers as K as
%"an operator" eq 2.108 and 2.112 in your thesis. Also shouldn't the $\text{d} \Gamma$
%be $\text{d} \Gamma'$? }

%\begin{equation}\label{eq:single_layer}
%V^{\mathbf{r}_\Gamma}_L (\psi(\mathbf{r}_\Gamma)) = \oint_\Gamma \psi(\mathbf{r}'_\Gamma) G_L(\mathbf{r}_\Gamma, \mathbf{r}'_\Gamma) \text{d} \Gamma.
%\end{equation}

%\begin{equation}\label{eq:double_layer}
%K^{\mathbf{r}_\Gamma}_L (\psi(\mathbf{r}_\Gamma)) = \oint_\Gamma \psi(\mathbf{r}'_\Gamma) \frac{\partial}{\partial \mathbf{n}}G_L(\mathbf{r}_\Gamma, \mathbf{r}'_\Gamma) \text{d} \Gamma.
%\end{equation}

%We can rewrite equation \eqref{eq:lap_bie} using the operator notation, as:

%\begin{equation} \label{eq:lap_operator}
%\left[ \frac{\mathbb{I}}{2} + K_L^{\mathbf{r}_\Gamma} \right] \left( \phi_\Gamma \right) = V_L^{\mathbf{r}_\Gamma} \left( \frac{\partial}{\partial \mathbf{n}} \phi_\Gamma \right),
%\end{equation}

%\noindent where $\mathbb{I}$ is the identity operator.

%Following the same steps, we can rewrite the Laplace equations in equation
%\eqref{eq:electrostatic_scatter} as:

%
%\begin{align} \label{eq:integral_eq_lspr_nobc}
%\frac{\phi_{1s,\Gamma}}{2}+ K_{L}^{\Gamma}(\phi_{1s,\Gamma}) - V_{L}^{\Gamma} \left(\frac{\partial}{\partial \mathbf{n}}\phi_{1s,\Gamma} \right) = 0&  \nonumber \\
%\frac{\phi_{2s,\Gamma}}{2} - K_{L}^{\Gamma}(\phi_{2s,\Gamma}) + V_{L}^{\Gamma} \left( \frac{\partial}{\partial \mathbf{n}} \phi_{2s,\Gamma} \right) = 0& \quad \text{on $\Gamma$,}
%\end{align}

%Applying the interface conditions of Equation \eqref{eq:electrostatic_scatter},
%we get:

%\begin{align} \label{eq:integral_eq_lspr}
%\frac{\phi_{1s,\Gamma}}{2}+ K_{L}^{\Gamma}(\phi_{1s,\Gamma}) - V_{L}^{\Gamma} \left(\frac{\partial}{\partial \mathbf{n}}\phi_{1s,\Gamma} \right) = 0&  \nonumber \\
%\frac{\phi_{1s,\Gamma}}{2} - K_{L}^{\Gamma}(\phi_{1s,\Gamma}) + \frac{1}{\epsilon_2}V_{L}^{\Gamma} \left( \epsilon_1 \frac{\partial}{\partial \mathbf{n}} \phi_{1s,\Gamma} - (\epsilon_2-\epsilon_1) \frac{\partial}{\partial \mathbf{n}} \phi_{i,\Gamma} \right) = 0& \quad \text{on $\Gamma$.}
%\end{align}

\newpage
\subsubsection{Analyte-sensor electrostatic potential under an electric field}


The sketch in Figure \ref{fig:analyte-sensor} shows a metallic nanoparticle ($\Omega_1$) interacting with an analyte ($\Omega_3$), under an external electric field.
Mathematically, this situation can be modeled as

\begin{align}\label{eq:electrostatic_scatter_prot_sen}
\nabla^2 \phi_{1s} &= 0, \qquad \nabla^2 \phi_{2s} = 0 \qquad\text{on $\Omega_1$, $\Omega_2$} \nonumber\\
\nabla^2 \phi_{3s} &= -\frac{1}{\epsilon_3} \sum_{k=0}^{N_q} \delta(|\mathbf{r}-\mathbf{r}_k|) q_k \qquad\text{on $\Omega_3$} \nonumber \\
\epsilon_1\frac{\partial\phi_{1s}}{\partial \mathbf{n}} - \epsilon_2\frac{\partial\phi_{2s}}{\partial\mathbf{n}} &= (\epsilon_2-\epsilon_1)\frac{\partial\phi_i}{\partial\mathbf{n}} \quad \phi_{1s} = \phi_{2s} \quad \text{on $\Gamma_1$}. \nonumber\\
\epsilon_3\frac{\partial\phi_{3s}}{\partial \mathbf{n}} - \epsilon_2\frac{\partial\phi_{2s}}{\partial\mathbf{n}} &= (\epsilon_2-\epsilon_3)\frac{\partial\phi_i}{\partial\mathbf{n}} \quad \phi_{3s} = \phi_{2s} \quad \text{on $\Gamma_2$}.
\end{align}
%
where $q_k$ are the point charges of the atoms inside the protein, located at $\mathbf{r}_k$.

\paragraph{Integral formulation}

Similar to Equation \eqref{eq:integral_eq_lspr}, we can write the system of partial differential equations in \eqref{eq:electrostatic_scatter_prot_sen} as


\begin{widetext} % to get the full-pagewidth equations in the two-column layout


\begin{align} \label{eq:integral_eq_lspr_nobc_system}
\frac{\phi_{1s,\Gamma_1}}{2}+ K_{L,\Gamma_1}^{\Gamma_1}(\phi_{1s,\Gamma_1}) - V_{L,\Gamma_1}^{\Gamma_1} \left(\frac{\partial}{\partial \mathbf{n}}\phi_{1s,\Gamma_1} \right) &= 0  \nonumber \\
\frac{\phi_{2s,\Gamma_1}}{2} - K_{L,\Gamma_1}^{\Gamma_1}(\phi_{2s,\Gamma_1}) + V_{L,\Gamma_1}^{\Gamma_1} \left(\frac{\partial}{\partial \mathbf{n}}\phi_{2s,\Gamma_1} \right) 
 - K_{L,\Gamma_2}^{\Gamma_1}(\phi_{2s,\Gamma_2}) + V_{L,\Gamma_2}^{\Gamma_1} \left(\frac{\partial}{\partial \mathbf{n}}\phi_{2s,\Gamma_2} \right) &= 0  \nonumber \\
\frac{\phi_{2s,\Gamma_2}}{2} - K_{L,\Gamma_1}^{\Gamma_2}(\phi_{2s,\Gamma_1}) + V_{L,\Gamma_1}^{\Gamma_2} \left(\frac{\partial}{\partial \mathbf{n}}\phi_{2s,\Gamma_1} \right)  
- K_{L,\Gamma_2}^{\Gamma_2}(\phi_{2s,\Gamma_2}) + V_{L,\Gamma_2}^{\Gamma_2} \left(\frac{\partial}{\partial \mathbf{n}}\phi_{2s,\Gamma_2} \right) &= 0  \nonumber \\
\frac{\phi_{3s,\Gamma_2}}{2} + K_{L,\Gamma_2}^{\Gamma_2}(\phi_{3s,\Gamma_2}) - V_{L,\Gamma_2}^{\Gamma_2} \left( \frac{\partial}{\partial \mathbf{n}} \phi_{3s,\Gamma_2} \right) &= \frac{1}{4\pi\epsilon_3} \sum_{k=0}^{N_q} \frac{q_k}{|\mathbf{r}_{\Gamma_2} - \mathbf{r}_k|} ,
\end{align}
%
\noindent where $V$ and $K$ are the single- and double-layer operators in equations 
\eqref{eq:single_layer} and \eqref{eq:double_layer}. In this case, we distinguish between the
surface where the integrals run (subindex), and the surface that contains the evaluation point (superindex).

Applying the interface condition of equation \eqref{eq:electrostatic_scatter_prot_sen},
leads to: 


\begin{align} \label{eq:integral_eq_lspr_system}
\frac{\phi_{1s,\Gamma_1}}{2}&+ K_{L,\Gamma_1}^{\Gamma_1}(\phi_{1s,\Gamma_1}) - V_{L,\Gamma_1}^{\Gamma_1} \left(\frac{\partial}{\partial \mathbf{n}}\phi_{1s,\Gamma_1} \right) = 0  \nonumber \\
 \frac{\phi_{1s,\Gamma_1}}{2}& - K_{L,\Gamma_1}^{\Gamma_1}(\phi_{1s,\Gamma_1}) + V_{L,\Gamma_1}^{\Gamma_1} \left(\frac{\epsilon_1}{\epsilon_2}\frac{\partial}{\partial \mathbf{n}}\phi_{1s,\Gamma_1} \right) - V_{L,\Gamma_1}^{\Gamma_1} \left(\frac{\epsilon_2-\epsilon_1}{\epsilon_2}\frac{\partial}{\partial \mathbf{n}}\phi_{i,\Gamma_1} \right) \nonumber\\ 
 & - K_{L,\Gamma_2}^{\Gamma_1}(\phi_{3s,\Gamma_2}) + V_{L,\Gamma_2}^{\Gamma_1} \left(\frac{\epsilon_3}{\epsilon_2}\frac{\partial}{\partial \mathbf{n}}\phi_{3s,\Gamma_2} \right)  - V_{L,\Gamma_2}^{\Gamma_1} \left(\frac{\epsilon_2 -\epsilon_3}{\epsilon_2}\frac{\partial}{\partial \mathbf{n}}\phi_{i,\Gamma_2} \right) = 0   \nonumber \\
 \frac{\phi_{3s,\Gamma_1}}{2}& - K_{L,\Gamma_1}^{\Gamma_2}(\phi_{1s,\Gamma_1}) + V_{L,\Gamma_1}^{\Gamma_2} \left(\frac{\epsilon_1}{\epsilon_2}\frac{\partial}{\partial \mathbf{n}}\phi_{1s,\Gamma_1} \right) - V_{L,\Gamma_1}^{\Gamma_2} \left(\frac{\epsilon_2-\epsilon_1}{\epsilon_2}\frac{\partial}{\partial \mathbf{n}}\phi_{i,\Gamma_1} \right) \nonumber \\
& - K_{L,\Gamma_2}^{\Gamma_2}(\phi_{3s,\Gamma_2}) + V_{L,\Gamma_2}^{\Gamma_2} \left(\frac{\epsilon_3}{\epsilon_2}\frac{\partial}{\partial \mathbf{n}}\phi_{3s,\Gamma_2} \right)  - V_{L,\Gamma_2}^{\Gamma_2} \left(\frac{\epsilon_2 -\epsilon_3}{\epsilon_2}\frac{\partial}{\partial \mathbf{n}}\phi_{i,\Gamma_2} \right) = 0  \nonumber \\
\frac{\phi_{3s,\Gamma_2}}{2}& + K_{L,\Gamma_2}^{\Gamma_2}(\phi_{3s,\Gamma_2}) - V_{L,\Gamma_2}^{\Gamma_2} \left( \frac{\partial}{\partial \mathbf{n}} \phi_{3s,\Gamma_2} \right) = \frac{1}{4\pi\epsilon_3} \sum_{k=0}^{N_q} \frac{q_k}{|\mathbf{r}_{\Gamma_2} - \mathbf{r}_k|} 
\end{align}
\end{widetext}


\begin{figure}%[b] %  figure placement: here, top, bottom, or page
    \centering
    \includegraphics[width=0.25\textwidth]{protein_sensor_regions.pdf} 
    \caption{Analyte-sensor system under electric field.}
    \label{fig:analyte-sensor}
 \end{figure}

\paragraph{Discretization and linear system}

We discretize the surface into flat triangles, and assume that  $\phi$ and 
$\partial \phi/\partial \mathbf{n}$ are constant within each element. We can
then write the layer operators in their discretized form as follows:
%
\begin{align} \label{eq:layers_disc}
V_{L,\text{disc}}^{\mathbf{r}_\Gamma} \left( \frac{\partial}{\partial \mathbf{n}} \phi(\mathbf{r}_{\Gamma}) \right) &= \sum_{j=1}^{N_p} \frac{\partial}{\partial \mathbf{n}} \phi(\mathbf{r}_{\Gamma_j}) \int_{\Gamma_j} G_L(\mathbf{r}_\Gamma,\mathbf{r}_{\Gamma_j})  \mathrm{d} \Gamma_j  \nonumber \\
K_{L,\text{disc}}^{\mathbf{r}_\Gamma}(\phi(\mathbf{r}_{\Gamma})) &=  \sum_{j=1}^{N_p}\phi(\mathbf{r}_{\Gamma_j})\int_{\Gamma_j} \frac{\partial}{\partial \mathbf{n}} \left[ G_L(\mathbf{r}_\Gamma,\mathbf{r}_{\Gamma_j}) \right]\mathrm{d} \Gamma_j
\end{align}
%
\noindent where $N_p$ is the number of discretization elements on $\Gamma$, 
and $\phi(\mathbf{r}_{\Gamma_j})$ and $\frac{\partial}{\partial \mathbf{n}} 
\phi(\mathbf{r}_{\Gamma_j})$ are the values of $\phi$ and 
$\frac{\partial \phi}{\partial \mathbf{n}}$ on panel $\Gamma_j$.
Using centroid collocation, we can write equation \eqref{eq:integral_eq_lspr} in matrix form as:
%
 \begin{equation} \label{eq:matrix_lspr}
 \left[
    \begin{matrix} 
       \frac{1}{2} + K_{L}^{\Gamma} & -V_{L}^{\Gamma}  \vspace{0.2cm} \\
       \frac{1}{2} - K_{L}^{\Gamma} &  \frac{\epsilon_1}{\epsilon_2} V_{L}^{\Gamma}  \vspace{0.2cm} 
    \end{matrix}
    \right] \left[ 
    \begin{matrix} 
       \phi_{1s,\Gamma} \vspace{0.2cm} \\
       \frac{\partial}{\partial \mathbf{n}} \phi_{1s,\Gamma} \vspace{0.2cm}
    \end{matrix} 
     \right] =   
    \left[
    \begin{matrix} 
       0 \\
       V_{L}^{\Gamma} \left(\frac{\epsilon_2-\epsilon_1}{\epsilon_2}\right) \frac{\partial\phi_i}{\partial\mathbf{n}} \vspace{0.2cm} 
    \end{matrix}
    \right]
 \end{equation}
%
Equation \eqref{eq:integral_eq_lspr_system} can be represented as:
%
\begin{align} \label{eq:matrix_multi}
 \left[
    \begin{matrix} 
       \frac{1}{2}+K_{L, \Gamma_1}^{\Gamma_1} & -V_{L, \Gamma_1}^{\Gamma_1} & 0 &  0   \vspace{0.2cm} \\
       \frac{1}{2}-K_{L, \Gamma_1}^{\Gamma_1} & \frac{\epsilon_1}{\epsilon_2} V_{L, \Gamma_1}^{\Gamma_1} & -K_{L, \Gamma_2}^{\Gamma_1} & \frac{\epsilon_3}{\epsilon_2} V_{L, \Gamma_2}^{\Gamma_1} \vspace{0.2cm}  \\
        -K_{L, \Gamma_1}^{\Gamma_2}&\frac{\epsilon_1}{\epsilon_2} V_{L, \Gamma_1}^{\Gamma_2} & \frac{1}{2}-K_{L, \Gamma_2}^{\Gamma_2}  &  \frac{\epsilon_3}{\epsilon_2} V_{L, \Gamma_2}^{\Gamma_2} \vspace{0.2cm} \\
       0 & 0 & \frac{1}{2}+K_{L, \Gamma_2}^{\Gamma_2}&  - V_{L, \Gamma_2}^{\Gamma_2}   \vspace{0.2cm} \\
    \end{matrix}
    \right] 
\cdot
 \left[
    \begin{matrix}
    \phi_{1,\Gamma_1} \vspace{0.2cm} \\
    \frac{\partial}{\partial \mathbf{n}} \phi_{1,\Gamma_1} \vspace{0.2cm} \\
    \phi_{3,\Gamma_2} \vspace{0.2cm} \\
    \frac{\partial}{\partial \mathbf{n}} \phi_{3,\Gamma_2} \vspace{0.2cm} \\
    \end{matrix}
\right]&
 \nonumber \\
 = \left[
    \begin{matrix}
    0 \vspace{0.2cm} \\
    V_{L,\Gamma_1}^{\Gamma_1} \left(\frac{\epsilon_2-\epsilon_1}{\epsilon_2}\frac{\partial}{\partial \mathbf{n}}\phi_{i,\Gamma_1} \right)
    + V_{L,\Gamma_2}^{\Gamma_1} \left(\frac{\epsilon_2 -\epsilon_3}{\epsilon_2}\frac{\partial}{\partial \mathbf{n}}\phi_{i,\Gamma_2} \right)
    \vspace{0.2cm}\\
    V_{L,\Gamma_1}^{\Gamma_2} \left(\frac{\epsilon_2-\epsilon_1}{\epsilon_2}\frac{\partial}{\partial \mathbf{n}}\phi_{i,\Gamma_1} \right)
    + V_{L,\Gamma_2}^{\Gamma_2} \left(\frac{\epsilon_2 -\epsilon_3}{\epsilon_2}\frac{\partial}{\partial \mathbf{n}}\phi_{i,\Gamma_2} \right)
    \vspace{0.2cm}\\
    \frac{1}{4\pi\epsilon_3}\sum_{k=0}^{N_q} \frac{q_k}{|\mathbf{r}_{\Gamma_2} - \mathbf{r}_k|} \vspace{0.2cm}  \\
    \end{matrix}
\right]&
\end{align}
%
\noindent where the elements of the matrix are
%
\begin{align} \label{eq:layers_element}
V_{L,ij}^{\Gamma} &= \int_{\Gamma_j} G_L(\mathbf{r}_{\Gamma_i},\mathbf{r}_{\Gamma_j})  \mathrm{d} \Gamma_j, \nonumber \\
K_{L,ij}^{\Gamma} &= \int_{\Gamma_j} \frac{\partial}{\partial \mathbf{n}} \left[ G_L(\mathbf{r}_{\Gamma_i},\mathbf{r}_{\Gamma_j}) \right]\mathrm{d} \Gamma_j,
\end{align}
%
\noindent with $\mathbf{r}_{\Gamma_i}$ being at the center of panel $\Gamma_i$.


\paragraph{Integral evaluation}

We evaluate the integrals in Equation \eqref{eq:layers_element} with Gauss quadrature
rules. The $1/r$ singularity of the Green's function poses a
problem to obtaining good accuracy when the integral is 
singular or near-singular. Therefore, we define three different regions, as follows.
\begin{description}
\item[Singular integrals:] If the collocation point is in the integration element,
the singularity is difficult to resolve with standard
Gauss integration schemes. In this case, we use a semi-analytical technique 
\cite{HessSmith1967,ZhuHuangSongWhite2001} that places $N_k$ quadrature nodes on the 
edges of the triangle.

\item[Near-singular integrals:] If the collocation point is close to the integration element,
the integrand has a high gradient, and high-order quadrature rules are required. 
We use the representative length of the integrated triangle ($L = \sqrt{2\cdot\text{Area}}$)
to define a threshold of the \emph{nearby} region, for example, when the integration panel 
is $2L$ or less away from the collocation point. For near-singular integrals, we use 
$K_{fine}=19, 25 \; or \; 37$ points per triangle. 

\item[Far-away integrals:] When the distance between the collocation point and the integration
element is beyond the threshold, they are considered to be far-away. 
At this point, the integrand is smooth enough that we obtain good 
accuracy with low-order integration, for example, with 
$K=1, 3 \; or \; 4$ Gauss quadrature points per boundary element. 
\end{description}

\subsubsection{Boundary integral expression of the dipole moment}

As shown in Equation \eqref{eq:scat_efield_long_range}, the scattered electric 
field in the far-away limit depends on the dipole moment. The dipole moment is 
defined as 
%
\begin{equation} \label{eq:dipole_def}
\mathbf{p} = \int_\Omega \mathbf{r} \rho \text{d}\Omega,
\end{equation}
%
and rewriting this equation using Gauss' law, we obtain
%
\begin{equation} \label{eq:dipole_def_gauss}
\mathbf{p} = -\epsilon_2\int_\Omega \mathbf{r} \nabla^2 \phi_{2s} \text{d}\Omega.
\end{equation}
%
For component $i$, this becomes:
%
\begin{equation} \label{eq:dipole_def_gauss_i}
{p_i} = -\epsilon_2\int_\Omega {x_i} \nabla^2 \phi_{2s} \text{d}\Omega.
\end{equation}
%
Using the identity
%
\begin{equation} \label{eq:identity_grad}
  \nabla \cdot \left(f \mathbf{v}\right) = \left( \nabla f \right)\cdot \mathbf{v} + f\left(\nabla \cdot \mathbf{v}\right)
\end{equation}
%
with $f=x_i$ and $\mathbf{v} = \nabla\phi_{2s}$, we can rewrite Equation \eqref{eq:dipole_def_gauss_i}
as 
%
\begin{equation}
- \frac{p_i}{\epsilon_2} = \int_\Omega \nabla \cdot \left( x_i \nabla \phi_{2s} \right) \; \text{d}\Omega - \int_\Omega \nabla x_i \cdot \nabla\phi_{2s} \; \text{d}\Omega, \nonumber 
\end{equation}
\noindent and applying the divergence theorem
\begin{equation} \label{eq:dip_gauss_interm_1}
- \frac{p_i}{\epsilon_2}= \oint_\Gamma  x_i  \nabla \phi_{2s} \cdot \mathbf{n} \; \text{d}\Gamma - \int_\Omega \nabla x_i \cdot \nabla\phi_{2s} \; \text{d}\Omega.
\end{equation}
%
Using the identity \eqref{eq:identity_grad} again in Equation \eqref{eq:dip_gauss_interm_1}, this time 
taking $f=\phi_{2s}$ and $\mathbf{v} = \nabla x_i$, we get:
%
\begin{align} \label{eq:dip_gauss_interm_2}
 - \frac{p_i}{\epsilon_2} =& \oint_\Gamma  x_i  \frac{\partial \phi_{2s}}{\partial \mathbf{n}} \text{d}\Gamma - \nonumber \\
 & \left[ \int_\Omega \nabla \cdot \left( \phi_{2s} \nabla x_i \right)\;\text{d}\Omega - \int_\Omega  \phi_{2s} \nabla^2 x_i \;\text{d}\Omega\right] \nonumber\\
%&\text{and applying the divergence theorem} \nonumber \\
=& \oint_\Gamma  x_i  \frac{\partial \phi_{2s}}{\partial \mathbf{n}} \; \text{d}\Gamma - \oint_\Gamma \phi_{2s} \nabla x_i \cdot \mathbf{n} \; \text{d}\Gamma \nonumber \\
=& \oint_\Gamma  x_i  \frac{\partial \phi_{2s}}{\partial \mathbf{n}} \; \text{d}\Gamma - \oint_\Gamma \phi_{2s} n_i \;\text{d}\Gamma
\end{align}
%
Throughout this derivation, the normals are pointing into $\Omega_1$. However, in our implementation 
all normals are pointing outwards, and we need to include an extra negative sign, yielding:
%
\begin{equation} \label{eq:dipole_def_gauss_i_final}
{p_i} = \epsilon_2 \left[ \oint_\Gamma  x_i  \frac{\partial \phi_{2s}}{\partial \mathbf{n}} \text{d}\Gamma - \oint_\Gamma \phi_{2s} n_i \; \text{d}\Gamma \right].
\end{equation}

Using BEM, we obtain the electrostatic potential on the surface of the nanoparticle, 
which we use in Equation \eqref{eq:dipole_def_gauss_i_final} to get the dipole 
moment, and in Equation \eqref{eq:scat_efield_long_range} to obtain the scattered
electric field. We can then use Equation \eqref{eq:scat_efield_fwa} and Equation 
\eqref{eq:cext_fwa} to get the extinction cross section.

\subsection{Acceleration strategies} \label{sec:acc_strategies}

One disadvantage of the Boundary Element Method (BEM) is that it generates dense matrices
after discretization. Solving the resulting linear system using
Gaussian elimination would require $\O{N^3}$ computations and $\O{N^2}$ storage, whereas for a
Krylov-subspace iterative solver, like the Generalized Minimal Residual Method (GMRES),
computations drop to $\O{N^2}$ because they are dominated by dense matrix-vector 
products. This makes BEM inefficient with more than a few thousand boundary elements,
which are the mesh sizes required for real applications. 

In our formulation with Gaussian quadrature and collocation, the matrix-vector product
becomes an N-body problem, with Gauss nodes acting as centers of mass (\emph{sources}), 
and the collocation points acting as evaluation points for the potential (\emph{targets}).
To overcome the unfavorable scaling,
we accelerate the matrix-vector product using a treecode algorithm \cite{BarnesHut1986,DuanKrasny2001}, 
which is a fast-summation algorithm capable of reducing $\O{N^2}$
computational patterns like
%
\begin{equation} \label{eq:summation}
V(\mathbf{x}_i) = \sum_{j=1}^{N} q_j \psi(\mathbf{x}_i, \mathbf{y}_j) 
\end{equation}
%
\noindent to a computational complexity of $\O{N \log N}$. In Equation \eqref{eq:summation} 
$q_j$ is the weight, $\psi$ the kernel, $\mathbf{y}_j$ the locations of sources and 
$\mathbf{x}_i$ the locations of targets.

The treecode groups sources geometrically in boxes of an octree, built ensuring
that no box in the lowest level has more than $N_\text{crit}$ sources. If a group of
sources is far away from a target, their influence is aggregated at an expansion center,
and the target interacts with the box, rather than with each source independently.
If the group of targets is  close, the treecode queries the child
boxes. If the box has no children and still is not far enough, the interaction is 
performed directly via \eqref{eq:summation}.
 The threshold to decide if a box is far enough is called the multipole-
acceptance criterion (MAC), defined as:
%
\begin{equation}
\theta > \frac{r_b}{r},
\end{equation}
%
\noindent where $r_b$ is the box size and $r$ the distance between the box center and the target.
Common values of $\theta$ are $1/2$ and $2/3$.
To approximate the contribution of the sources, we use Taylor expansions
of order $P$.
The treecode allows us to control the accuracy of the approximation by modifying $\theta$ and $P$.
Further details of the Treecode implementation in \pygbe can be found in \cite{CooperBarba-share154331,CooperBardhanBarba2013}.

\subsection{Code modifications and added features} \label{sec:code_imp}

As mentioned at the beginning of this section, the present work extends the \pygbe code
to allow its application to nano-plasmonics. 
The code required the following modifications and added features:

\begin{itemize}
    \item Re-writing the GMRES solver to accept complex numbers. 
    \item Splitting treecode calculations into real and imaginary parts.
    \item Re-formatting configuration files to includes electric field intensity and  wavelength.
    \item Adding the new function \texttt{read\_electric\_field}, to read the electric field intensity and its wavelength from configuration files.
    \item Adding the new function \texttt{dipole\_moment} to compute numerically the dipole moment by Equation \eqref{eq:dipole_def_gauss_i_final}.
    \item Adding the new function \texttt{extinction\_cross\_section} to compute the the extinction cross section.
    \item Organizing LSPR computations on a different main script (called \texttt{lspr.py}).
\end{itemize}

For information about how to use the code, run examples and tests, see the
\pygbe documentation at \url{http://barbagroup.github.io/pygbe/docs/}

\subsection{Protein mesh preparation}
In Figure \ref{fig:analyte-sensor}, $\Omega_3$ is a region that represents the analyte molecule, which contains a point charge distribution of the partial charges, and is interfaced with the solvent by $\Gamma_2$, the solvent excluded surface (\texttt{SES}).
The \texttt{SES} is generated by rolling a spherical probe of the size of a water molecule ($1.4$\AA~ radius) around the analyte, and tracking the points where the probe and molecule make contact.
The open-source software Nanoshaper \cite{Nanoshaper} uses the molecular structure to produce a triangulation of the \texttt{SES}, which can be read by our software.
In particular, Nanoshaper takes as inputs the atomic coordinates, obtained from the Protein Data Bank, and radii, which were 
extracted from a \texttt{pqr} file generated with \texttt{pdb2pqr} \cite{Dolinsky04}.
We obtained the charge and van der Waals parameters of the analyte from \texttt{pdb2pqr} using the built-in \texttt{amber} force field.

