%!TEX root = ClementiCooperBarba2018.tex

%* what is known
Localized surface plasmon resonance (LSPR) is an optical effect where an 
electromagnetic wave excites the free electrons on the surface of a metallic nanoparticle.
The vibrations of the electron cloud are known as plasmons, and in LSPR they resonate with the incoming
field (see Figure \ref{fig:lspr}). When this happens, most of the incoming energy is
either absorbed by the nanoparticle, or scattered in different directions, 
both creating a shadow behind the scatterer (a.k.a., extinction). It is worth mentioning
that for nanoparticles smaller than 20 nm absorption dominates over scattering, making scattering 
contributions negligible \cite{PetryayevaKrull2011, OlsonETal2015}. In LSPR, the 
wavelength of the incoming wave is often much larger than the size of the nanoparticle, 
which allows for valid approximations that simplify the mathematical model.

The phenomenon of LSPR can be used for biosensing, 
as the resonance frequency is highly dependent on the dielectric environment 
around the scatterer. 
The resonance frequency shifts whenever an analyte binds to the nanoparticle, 
resulting in a very sensitive means of detecting its presence \cite{HaesVanduyne2002,HaesETal2004}.

Numerical models for LSPR generally rely on the 
solution of Maxwell's equations in some form, using finite difference time-domain (FDTD),
boundary element, or finite element methods \cite{SolisTaboadaObelleiroLiz-MaarzanGarciadeabajo2014}. 
These methods have been used to study the 
optical properties of dielectric or metallic nanoparticles \cite{Hohenester2018,HohenesterTrugler2012,
JungPedersenSondergaardPedersenLarsenNielsen2010, VideenSun2003,
MayergoyzFredkinZhang2005, MayergoyzZhang2007}, interactions between nanoparticles
and electron beams \cite{GarciadeabajoAizpurua1997, GarciadeabajoHowie2002},
and surface plasmon resonance sensors.
In the latter application, researchers have used simple mathematical models for the 
interaction between a metallic nanoparticle and biomolecules,
like representing the medium and the dissolved analytes with an effective permittivity \cite{JungCampbellChinowskyMarYee1998,WilletsVandyune2007,PhanETal2013}, 
or representing the target molecules as spheres 
\cite{DavisGomezVernon2010,AntosiewiczApellClaudioKall2011}.


\begin{figure}
 \centering
   \includegraphics[width=0.35\textwidth]{lspr.pdf} 
   \caption{Illustration of the localized surface plasmon resonance (LSPR) effect of a metallic nanoparticle under an electromagnetic field.    \label{fig:lspr}}
  \end{figure}

   
%* What is unknown, limitations and gaps
Progress in biosensor research is still predominantly made
through experimental investigations, which can often be costly and time consuming.
Computational approaches could assist the design process and play a role  
in optimizing biosensors, giving access to details that are not available in experimental settings.
For example, empirical studies showed that the sensitivity of the sensor
is highly dependent on the distance between the nanoparticle and the analyte \cite{HaesETal2004}.
These studies were complemented with models using a discrete dipole approximation (DDA),
which includes the effect of the analyte through the effective permittivity. 
Other experimental studies complemented by modeling fully ignore the presence of the target molecules.
For example, Beuwer et al.~\cite{BeuwervanHoofZijlstra2018} and Henkel et al.~\cite{HenkelETal2018} 
used a boundary element method (BEM) in studies of the sensitivity of plasmonic sensors 
relying on (at least) two metallic nanoparticles (one on the sensor and one attached to the analyte).
Explicitly including the target molecules in the model may be needed in some cases, however.
For instance, despite experimental evidence showing that LSPR sensors are sensitive enough to detect 
conformational changes of the analytes \cite{HallETal2011}, 
these simplified models are not able to capture such details.

%* Fill the gap

Even though LSPR is an optical effect, electrostatic theory
provides a good approximation in the long-wavelength limit. This work uses
the boundary integral electrostatics solver \pygbe \cite{CooperETal2016} 
to compute the extinction cross-section of metallic nanoparticles, and to study how LSPR 
response changes in the presence of a biomolecule. 
We treat Maxwell's equations quasi-statically \cite{MayergoyzZhang2007} and
explicitly represent the target biomolecules by a surface mesh built from the crystal structures. 

\pygbe is a Python implementation of continuum electrostatic theory, used
for computing solvation energy of biomolecular systems. 
It has also been used to study protein orientation near charged nanosurfaces \cite{CooperClementiBarba2015}.
The code was recently extended to allow for complex dielectric constants 
\cite{ClementiETal2017}, aiming towards the LSPR biosensing application. 
The boundary element solver in \pygbe
is accelerated algorithmically via a treecode---an $\mathcal{O}(N\log N)$ fast-summation method---and on hardware by taking advantage of graphic processing units (GPUs). 
With these features, \pygbe is able to easily handle problems with in the order of 
half a million boundary elements, or more, 
allowing for the explicit representation of the biomolecular surface.
Other research software that could be used in this setting includes  
BEM++ \cite{SmigajETal2015} and a Matlab toolbox called MNPBEM \cite{HohenesterTrugler2012}, which have the capability to solve the full Maxwell's equation and the electrostatic approximation in the long-wavelength limit.
We believe in both cases the size of problems that can be treated, in terms of number of boundary elements, may not be enough to resolve the details of target biomolecules from their crystal structure. 

The software is shared under the BSD 3-clause license 
and is openly developed via its repository on Github.\footnote{\url{https://github.com/barbagroup/pygbe}}
This study also follows careful reproducibility practices, and all materials necessary
to reproduce the results are publicly available in reproducibility packages.
We use the Figshare and Zenodo services to deposit the computational meshes,
input and configuration files, and file bundles corresponding to the main figures in the paper.
See the figure captions for references to the open data artifacts.

%{\color{red}  Keeping this structure here for reference until we polish better
%the introduction.

%What is known:
%\begin{itemize}
%\item Plasmonic simulations, what applications they cover, etc. (cite Matlab guy, also Garcia de Abajo, Jung+Pedersen, etc. Maybe we can even cite COMSOL here)
%\item LSPR: how does it work, what simulations are there in the literature. Talk about work from, for example, Davis or van Duyne. See page 50 of my thesis (end chapter 2).
%\end{itemize}

%What is unkown:
%\begin{itemize}
%\item all developments are trial and error
%\item no computational tools to help in the design process
%\item example: no understanding of the sensitivity of the system
%\item models that consider the nanoparticle and the analyte are extremely simplistic
%\end{itemize}

%Fill the gap:
%\begin{itemize}
%item design of a computational tool that is highly accurate to represent the biomolecule
%\end{itemize}
%}

